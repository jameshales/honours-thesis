\chapter{Conclusions}

In this thesis we have provided sound and complete axiomatisations for the
single-agent refinement quantified epistemic and doxastic logics, and for the
multi-agent refinement quantified modal and doxastic logics. The completeness
proofs for each of these axiomatisations is performed via a provably correct
translation from the refinement quantified modal logics to the basic modal
logics, thus showing that each of the refinement quantified versions of the
logics we have considered are expressively equivalent to the basic modal
versions. This allows us to prove a number of properties about these refinement
quantified modal logics from the properties of the corresponding basic modal
logics, including decidability and the finite model property, and allows us to
derive model-checking and decision procedures for these logics via the
translations.

The logics presented in this thesis do not lend themselves directly to practical
applications, however results in logics based on those we have considered may
conceivably see practical uses in formal verification of software in the future.
The logics that we have considered quantify over all possible informative
updates, whilst practical applications often call for restrictions on the
informative updates that are possible. For example, in the setting of a game,
the moves that can and cannot be made are dictated by the rules of the game, or
in the setting of a security protocol, the communication that occurs is dictated
by the protocol, and thus the information that may be gleaned from these moves
or communication is accordingly restricted.

Furthermore, the notion of informative updates captured by refinements is better
suited to epistemic settings than to general modal or doxastic settings, and so
the omission of a result in the multi-agent epistemic setting is significant.
Although refinements can correspond to notions other than informative updates,
the correspondence to informative updates is our primary motivation.  Extensions
of these logics to other settings, such as to epistemic logic, may borrow from
the techniques used in the results we have presented in this thesis,
particularly the cover logic normal forms and the technique of a provably
correct translation to basic modal logics. However whether these techniques are
applicable or not depend on the setting that we are considering; for example van
Ditmarsch, Pinchinat and French~\cite{french2010future} have already
demonstrated that adding refinement quantifiers to the modal logic {\bf K4}
results in a logic that is strictly more expressive, and thus the technique of a
provably correct translation to the corresponding basic modal logic is not
feasible.

There are several immediate avenues for future work based on the results
presented in this thesis. We have not considered efficient decision or
model-checking procedures, succinctness results, or the effects of adding common
knowledge operators to the logics. Also we have already mentioned that results
in the multi-agent refinement quantified epistemic logic have yet to be given.

The provably correct translation used in the completeness proofs can be used to
derive a decision procedure for the logics we have considered. However, we note
that such a decision procedure would have a non-elementary time complexity. This
complexity is due to the translation to the various normal forms that we have
considered. Conversion to prenex normal form, disjunctive normal form, or
alternating disjunctive normal form results in a formula that is exponentially
larger than the original in the worst-case. As this conversion must be performed
for each refinement quantifier in the given formula, a formula with nested
refinement quantifiers results in an exponential increase in formula size for each
nested quantifier, thus the overall size of the resulting formula is
non-elementary in the size of the original formula. A more efficient decision
procedure would be desirable.

A decision procedure based on a tableau method was provided by van Ditmarsch,
French and Pinchinat~\cite{french2010future} for the single-agent refinement
quantified modal logic, that runs in 2EXP time in the worst case. A modification
of this tableau method may be applicable to the multi-agent refinement
quantified epistemic and doxastic logics, and is likely to perform better than
the non-elementary translation that we already have. A decision procedure for
the single-agent refinement quantified epistemic and doxastics logics that
runs in 2EXP time has already been presented by Hales, French and
Davies~\cite{hales2011refinement}.

Efficient model-checking procedures for these logics have yet to be considered.
Model-checking procedures can easily be derived from any decision procedure that
we may have, however it is often the case that there are model-checking
procedures for a logic that are more efficient than the corresponding decision
procedures. It is unknown whether there are model-checking procedures for any of
the logics considered so far, that are more efficient than the decision
procedures.

The provably correct translation used in the completeness proofs can also be
used to show that the refinement quantified modal logics we have considered are
expressively equivalent to their corresponding basic modal logics, so we raise
the question of whether the refinement quantified modal logics are more succinct
than the basic modal logics. It was shown by van Ditmarsch, French and
Pinchinat~\cite{french2010future} that the multi-agent refinement quantified
modal logic is exponentially more succinct than the basic modal logic.
Succinctness results for the refinement quantified doxastic and epistemic logics
have yet to be considered.

We may also wish to consider the addition of common knowledge operators to the
refinement quantified modal logics. In an epsitemic setting, common knowledge
refers to a situation where a group of agents knows some fact $\phi$, and every
agent in that group knows that the group knows that $\phi$, and so on ad
infinitum. A common question in dynamic epistemic logic is whether certain
knowledge can become common knowledge for a group of agents; common knowledge
tends to be considerably more difficult for agents to obtain than normal
knowledge. Common knowledge operators are used to signify that a group of agents
holds certain knowledge as common knowledge. Combined with refinement
quantifiers, this would allow us to pose questions such as whether or not
certain common knowledge is attainable.

We finally remark on the challenges involved in an axiomatisation for the
multi-agent refinement quantified epistemic logic. We can recognise the
axiomatisations for the multi-agent refinement quantified modal and doxastic
logics as generalisations of the axiomatisations for the single-agent variants
of these logics. In each case, we generalise the {\bf RK} or {\bf RKD45} axiom,
and introduce additional axioms, {\bf RComm} and {\bf RDist} to deal with the
interactions between different agents in the logics. These axiomatisations have
been designed so that the proof strategy of a provably correct translation is
feasible in order to show that the axiomatisations are complete. Consequently
the soundness proofs for these axioms have been more challenging than the
completeness proofs.

We have briefly considered an axiomatisation for the multi-agent refinement
quantified epistemic logic. The obvious generalisations however do not appear to
be sound. In the cases for \classK{} and \classKD{}, the soundness proofs for
our new axioms involve taking refinements in successor states, each of which
satisfy a particular formula, and combining these refinements into a larger
model where the corresponding successor states continue to satisfy the
corresponding formulae. The main obstacle in the epistemic setting is that we do
not yet have a result that allows us to safely combine such refinements into a
larger model whilst the successor states continue to satisfy the formulae we
require them to.  In the case of \classK{}, this was trivial, as we combine the
refinements using inward edges only, thus the successor states are bisimilar to
the refinements they were constructed from, and thus satisfy the same formulae.
In the case of \classKD{}, we relied on the properties of alternating
disjunctive normal formulae, in particular Lemma~\ref{kd45-successors}, which
allowed us to combine the refinements successfully.  Whilst every \logicS{}
formula is equivalent to a formula in alternating disjunctive normal form,
Lemma~\ref{kd45-successors} does not hold in \logicS{}. This is due to a
combination of the reflexivity and symmetric properties of \classS{} models. It
is because of these properties, particularly the symmetric property, that we
cannot expect models to continue to satisfy the formulae we require them to
satisfy when we combine them into larger models. 

A modification of the alternating disjunctive normal form may lead to a method
to safely combine the refinements in the manner we require, and thus give us a
provably correct translation. On the other hand, it may be the case that the
refinement quantified epistemic logic is more expressive than epistemic logic,
in which case the technique of a provably correct translation would be
infeasible. Previously Baltag, Solecki and Moss~\cite{baltag2004logics} gave a
provably correct translation from the action model logic to epistemic logic; we
note that if we had a provably correct translation from the refinement
quantified epistemic logic to epistemic logic, then we could combine these two
translations to give a translation from the arbitrary action model logic to
epistemic logic.
