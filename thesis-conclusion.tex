\chapter{Conclusions}

In this thesis we have provided sound and complete axiomatisations for the
single-agent refinement quantified epistemic and doxastic logics, and for the
multi-agent refinement quantified modal and doxastic logics. The completeness
proofs for each of these axiomatisations is performed via a provably correct
translation from the refinement quantified modal logics to the basic modal
logics, thus showing that each of the refinement quantified versions of these
logics are expressively equivalent to the basic modal versions. This allows us
to prove a number of properties about the refinement quantified modal logics,
from the properties of basic modal logics, including decidability and the finite
model property, and allows us to derive model-checking and decision procedures
for these logics via the translations.

Although the translation gives a decision procedure for the refinement
quantified modal logics, we note that the decision procedure has a
non-elementary time complexity. This is the case because of the conversion to
disjunctive normal forms or prenex normal forms in each of the translations.
This conversion must be performed for each refinement quantifier in the original
formula, and results in a worst-case exponential blow-up in the size of the
formula. This exponential blow-up is stacked for nested refinement operators,
and hence the time complexity of computing this translation is non-elementary.
van Ditmarsch, French and Pinchinat~\cite{french2010future} provided a decision
procedure for the single-agent refinement quantified modal logic that runs in
2EXP time, using a tableaux method, so it is worth considering a similar
approach for the refinement quantified modal logics we have considered here.
It is unknown whether there is a model-checking procedure for any of the
refinement quantified modal logics, that is more efficient than the decision
procedure.

van Ditmarsch, French and Pinchinat~\cite{french2010future} also showed
that the multi-agent refinement quantified modal logic is exponentially more
succinct than the basic modal logic, so we may also wish to consider the
succinctness of the refinement quantified epistemic and doxastic models.

We are also interested in an axiomatisation for the multi-agent refinement
quantified epistemic logic. We have considered an axiomatisation for the
multi-agent refinement quantified epistemic logic, but the obvious
generalisations from the single-agent case do not appear to work.

Finally we may wish to consider the addition of common knowledge operators to
the refinement quantified modal logics.
