\chapter{Conclusions}

In this thesis we have provided sound and complete axiomatisations for the
single-agent refinement quantified epistemic and doxastic logics, and for the
multi-agent refinement quantified modal and doxastic logics. The completeness
proofs for each of these axiomatisations is performed via a provably correct
translation from the refinement quantified modal logics to the basic modal
logics, thus showing that each of the refinement quantified versions of these
logics are expressively equivalent to the basic modal versions. This allows us
to prove a number of properties about the refinement quantified modal logics,
from the properties of basic modal logics, including decidability and the finite
model property, and allows us to derive model-checking and decision procedures
for these logics via the translations.

The results presented in this thesis do not lend themselves directly to
practical applications, however results in logics based on those we have
considered may conceivably see practical uses in the future. The logics that we
have considered quantify over all possible informative updates, whilst practical
applications often call for restrictions on the informative updates that are
possible. For example, in the setting of a game, the moves that can and cannot
be made are dictated by the rules of the game, or in the setting of a security
protocol, the communication that occurs is dictated by the protocol, and thus
the information that may be gleaned from these moves or communication is
accordingly restricted. Furthermore, the notion of informative updates captured
by refinements is better suited to epistemic settings than to general modal or
doxastic settings, and so the omission of a result in the multi-agent epistemic
setting is notable. However it is likely that extensions of these logics to
other settings may borrow from the techniques used in the results we have
presented in this thesis.

There are several immediate avenues for future work based on the results
presented in this thesis. We have not considered efficient decision or
model-checking procedures, succinctness results, or the effects of adding common
knowledge operators to the logics. Also we have already mentioned that results
in the multi-agent refinement quantified epistemic logic have yet to be given.

The provably correct translation used in the completeness proofs can be used to
derive a decision procedure for the logics we have considered. However, we note
that such a decision procedure would have a non-elementary time complexity. This
complexity is due to the translation to the various normal forms that we have
considered. Conversion to prenex normal form, disjunctive normal form, or
alternating disjunctive normal form results in a formula that is exponentially
larger than the original in the worst-case. As this conversion must be performed
for each refinement quantifier in the given formula, a formula with nested
refinement quantifiers results in an exponential increase in formula size for each
nested quantifier, thus the overall size of the resulting formula is
non-elementary in the size of the original formula. A more efficient decision
procedure would be desirable.

A decision procedure based on a tableau method was provided by van Ditmarsch,
French and Pinchinat~\cite{french2010future} for the single-agent refinement
quantified modal logic, that runs in 2EXP time in the worst case. A modification
of this tableau method may be applicable to the decision problem in the
multi-agent logics that we have considered here, and if it works it is likely to
perform better than the non-elementary procedure we already have.

Efficient model-checking procedures for these logics have yet to be considered.
Model-checking procedures can easily be derived from any decision procedure that
we may have, however it is often the case that there are model-checking
procedures for a logic that are more efficient than the corresponding decision
procedures. It is unknown whether there are model-checking procedures for any of
the logics considered so far, that are more efficient than the decision
procedures.

The provably correct translation used in the completeness proofs can also be
used to show that the refinement quantified modal logics we have considered are
expressively equivalent to their corresponding basic modal logics, so we raise
the question of whether the refinement quantified modal logics are more succinct
than the basic modal logics. It was shown by van Ditmarsch, French and
Pinchinat~\cite{french2010future} that the multi-agent refinement quantified
modal logic is exponentially more succinct than the basic modal logic.
Succinctness results for the refinement quantified doxastic and epistemic logics
have yet to be considered.

We are also interested in an axiomatisation for the multi-agent refinement
quantified epistemic logic. We have considered axiomatisations for this case
already, but the obvious generalisations from the single-agent case, or for the
multi-agent refinement quantified modal or doxastic logics, do not appear to
work.  A normal form, similar to the alternating disjunctive normal form used in
this thesis, may be useful for such an axiomatisation, as it was for the
multi-agent modal and doxastic cases.

Finally we may wish to consider the addition of common knowledge operators to
the refinement quantified modal logics. In an epistemic setting, common
knowledge refers to situations where a group of agents share some knowledge, and
are mutually aware that the other agents share this common knowledge. A common
question in dynamic epistemic logic is whether certain common knowledge is
attainable by a group of agents; common knowledge tends to be considerably more
difficult for agents to obtain than normal knowledge. A common knowledge
operator signifies that a collection of agents holds certain knowledge as common
knowledge. Combined with refinement quantifiers, this would allow us to pose
questions such as whether or not certain common knowledge is attainable.
