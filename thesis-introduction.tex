\chapter{Introduction}

% Introduce modal logic

{\em Modal logics} are used to reason intrinsically about the properties of
relational structures known as {\em Kripke models}, by qualifying the truth of
statements with operators known as {\em modalities}.  {\em Epistemic logic} is a
variant of modal logic, commonly known as the logic of {\em knowledge}, where
the Kripke models that are considered encode the knowledge that a collection of
agents hold about the state of the world. {\em Doxastic logic} is a similar
variant of modal logic, the logic of {\em belief}, which considers the beliefs
that a collection of agents hold.

% Introduce informative updates
{\em Dynamic epistemic logic} and {\em dynamic doxastic logic} are areas of
epistemic and doxastic logic which consider how the knowledge or beliefs of a
collection of agents may change in response to {\em informative updates}.
Informative updates are events, such as messages or announcements, which provide
agents with new information, whilst leaving the {\em factual} information about
the state of the world unchanged.  Dynamic epistemic logic and dynamic doxastic
logic consider questions such as what knowledge or belief states can result from
specific informative updates, and how or whether a specific knowledge or belief
state can result from arbitrary informative updates, perhaps with restrictions
on what kinds of informative updates are allowed.

% Muddy children example
A classic example of a problem in dynamic epistemic logic is that of the muddy
children puzzle, explained in detail by van Ditmarsch, van der Hoek and
Kooi~\cite{vanditmarsch2007dynamic} in their book:

\begin{quote}
A group of children has been playing outside and are called back into the house
by their father. The children gather round him. As one may imagine, some of them
have become dirty from the play and in particular: they may have mud on their
forehead. Children can only see whether other children are muddy, and not if
there is any mud on their own forehead. All this is commonly known, and the
children are, obviously, perfect logicians. Father now says: “At least one of
you has mud on his or her forehead.” And then: “Will those who know whether they
are muddy please step forward.” If nobody steps forward, father keeps repeating
the request.
\end{quote}

The puzzle is whether each of the children can eventually determine if they are
muddy or not, without communicating with one another, except by stepping forward
when they know whether they are muddy or not. It can be shown that, since the
children are perfect logicians, if there are $m$ muddy children, those children
will step forward after the father has asked his question $m$ times, and the
remaining children will step forward after the next time.

The muddy children puzzle is a classic example in dynamic epistemic logic
because it demonstrates how agents in an epistemic system can gain knowledge
indirectly from very restricted informative updates. In the case of $m$ muddy
children, for the first $m - 1$ times that the father asks his question, no
child steps forward, and so the only information that children are given each
time is that none of the other children knows whether or not they are muddy.
Despite being given the same, seemingly useless information, over and over, it
is enough for each child to eventually determine whether or not they are muddy,
and this fact is surprising.

% Motivate dynamic epistemic logic
Problems of this form motivates our research into dynamic epistemic logic and
dynamic doxastic logic; we wish to know whether it is possible for agents to
achieve certain knowledge or belief states after some sequence of informative
updates. % TODO - Applications of this!?


% Introduce informative updates
In the context of dynamic epistemic logic, notable representations of
informative updates include {\em public announcements} and {\em action models}.
Public announcements are a relatively simple representation of informative
updates, whilst action models are quite general, and are also cable of
expressing any public announcement. In the context of dynamic doxastic logic,
informative updates are often represented using the concept of {\em belief
revision}, however this concept varies considerably from the types of
informative updates discussed in this paper.

% Introduce previous logics
As we are interested in questions about what knowledge or belief states can
result from specific or arbitrary informative updates, we consider extensions of
epistemic or doxastic logics which introduce operators that allow us to reason
about the knowledge or belief states that result from such informative updates.
The {\em public announcement logic}, and the {\em action model logic} are
extensions of epistemic logic which introduce operators for reasoning about the
results of specific informative updates (in the form of public announcements and
action models, respectively). The {\em arbitrary public announcement logic} adds
to the public announcement logic an additional operator, for reasoning about the
results of arbitrary public announcements. This logic was shown to be
undecidable, so a similar logic for action models is also likely to be
undecidable, and so there has not been considerable research in this area.

% Introduce FEL
Instead, attention has been given to logics which include an operator for
reasoning about arbitrary informative updates, but which do not also have the
operators for reasoning about specific informative updates. The {\em future
event logic} (also known as {\em refinement quantified modal logic}) extends
epistemic logic with an operator for quantifying over {\em refinements} of the
current knowledge state, as represented by a Kripke model. The refinements of a
Kripke model can be shown to correspond to the results of arbitrary action
models applied to that Kripke model. The future event logic, as an extension of
single-agent modal logic, has previously been axiomatised and shown to be
decidable.

% The present work
The present work extends future event logic to the setting for which it was
originally intended: epistemic logic. We examine modal logic, doxastic logic and
epistemic logic, providing an axiomatisation for modal logic and doxastic logic
in the multi-agent cases, and providing a translation from epistemic logic to
doxastic logic, which gives rise to decidability results in the context of
epistemic logic.
