\chapter{Introduction}

% Introduce modal logic

{\em Modal logics} are extensions of propositional logic, used to reason about
the properties of relational structures known as {\em Kripke models}, by
qualifying the truth of statements with operators known as {\em modalities}.
{\em Epistemic logic} is a variant of modal logic, commonly known as the logic
of knowledge, where the Kripke models that are considered encode the knowledge
that a collection of agents hold about the state of the world, and the effect of
modalities is to say that an agent {\em knows} a statement to be true. {\em
Doxastic logic} is a similar variant of modal logic, the logic of belief, that
considers the beliefs that a collection of agents hold. The difference between
the notions of knowledge and belief used in epistemic and doxastic logics is
simply that anything that an agent knows must be true, whereas it is permissible
for an agent to believe something that is false.

% Introduce dynamic epistemic logic
{\em Dynamic epistemic logic} and {\em dynamic doxastic logic} are fields that
consider how the knowledge or beliefs of a collection of agents may change in
response to {\em informative updates}. Informative updates are events, such as
messages or announcements, that communicate new information to agents. 
Informative updates change the knowledge or beliefs of agents, whilst leaving
the {\em factual} information about the state of the world unchanged.  These
fields consider questions such as what knowledge or belief states can result
from specific informative updates, and how or whether a specific knowledge or
belief state can be achieved from a series of arbitrary informative updates,
perhaps with restrictions on what kinds of informative updates are allowed.

% Muddy children example
A classic example is that of the muddy children puzzle, explained by van
Ditmarsch, van der Hoek and Kooi~\cite{vanditmarsch2007dynamic}:

\begin{quote}
``A group of children has been playing outside and are called back into the house
by their father. The children gather round him. As one may imagine, some of them
have become dirty from the play and in particular: they may have mud on their
forehead. Children can only see whether other children are muddy, and not if
there is any mud on their own forehead. All this is commonly known, and the
children are, obviously, perfect logicians. Father now says: “At least one of
you has mud on his or her forehead.” And then: “Will those who know whether they
are muddy please step forward.” If nobody steps forward, father keeps repeating
the request.''
\end{quote}

The puzzle is whether each of the children can eventually determine whether
they are muddy or not. The children are not allowed to communicate with one
another, except by stepping forward when they know whether they are muddy or
not. It can be shown that if there are $m$ muddy children, those children will
step forward after the father has asked his question $m$ times, at which point
the remaining children will know that they are not muddy, and so they too will
step forward the next time the father asks his question.

The muddy children puzzle demonstrates how agents in an epistemic system can
gain knowledge about the state of the world indirectly. In the case of $m$ muddy
children, the first $m - 1$ times that the father asks his question, no child
steps forward, and so the only information that is given is that none of the
children knows whether or not they are muddy. Despite begin given the same
information over and over, that does not directly provide any information about
whether the children are muddy or not, this is enough for each child to
eventually determine whether or not they are muddy, and this fact is surprising.

% Motivate dynamic epistemic logic
Problems of this form motivates our research into dynamic epistemic logic and
dynamic doxastic logic; we wish to know whether it is possible for agents to
achieve certain knowledge or belief states after some sequence of informative
updates. This is an interesting question because, as we have already
demonstrated with the muddy children puzzle, it is not always obvious whether or
not certain knowledge or belief states are attainable. This has applications in
areas such as games with imperfect information, where one may wish to know
whether certain knowledge can be deduced from given information, or in security
protocols, where one may desire assurances that certain knowledge cannot be
indirectly deduced from seemingly innocuous communication.

% Introduce informative updates
Previous work has considered a number of different representations of
informative updates. In the context of epistemic logic, two of the most notable
are {\em public announcements}, and {\em action models}. Public announcements
are a relatively simple and restrictive form of informative update, whereas
actions models are more general and expressive. We avoid discussing
representations in doxastic logic for the time being, as the most notable,
belief revision, is conceptually different from the epistemic notions of
information change that the present work is based on.

% Motivate previous logics
As we are interested in questions about what knowledge or belief states can
result from informative updates, we consider extensions of epistemic or doxastic
logics that allow us to reason about this. There are two main directions that we
can take with such extensions. The first is logics with which we can reason
about the results of {\em specific} informative updates; in such a logic we can
say of some specific informative update that ``after the informative update
$\alpha$, the statement $\phi$ is true in the resulting knowledge state''. The
second is logics with which we can reason about the results of {\em arbitrary}
informative updates, by quantifying over them; in such a logic we can say
that ``there exists an informative update after which $\phi$ is true in the
resulting knowledge state''.

% Introduce PAL, AML, and APAL
The {\em public announcement logic}, introduced by
Plaza~\cite{plaza2007logics}, and also independently by Gerbrandy and
Groenvald~\cite{gerbrandy1997reasoning} is an extension of epistemic logic that
introduces an operator for reasoning about the results of a specific public
announcement. A similar logic, the {\em action model logic}, introduced by
Baltag and Moss~\cite{baltag2004logics}, allows reasoning about the results of
executing a specific action model. Balbiani, et
al.~\cite{balbiani2007arbitrary} then explored the {\em arbitrary public
announcement logic}, an extension of the public announcement logic that adds an
operator for reasoning about the results of arbitrary public announcements.  It
was later shown by van Ditmarsch and French~\cite{french2008undecidability}
that the arbitrary public announcement logic is undecidable. It has not been
considered whether a similar extension of action model logic would be decidable
or not, but the effect of the undecidability result in arbitrary public
announcement logic has been to encourage research into weaker versions of this
logic that are more likely to be decidable.

% Introduce FEL
The {\em future event logic}, introduced by van Ditmarsch and
French~\cite{french2009simulation} is an extension of epistemic logic that
introduces an operator for quantifying over the {\em refinements} of a Kripke
model. The finite refinements of a Kripke model can be shown to correspond to
the results of executing arbitrary action models on that Kripke model, and so
the effect of quantifying over the refinements of a Kripke model can reasonably
be said to be equivalent to quantifying over arbitrary informative updates. van
Ditmarsch, French and Pinchinat~\cite{french2010future} gave an axiomatisation
and decidability results for a simplified version of this logic, in the context
of single-agent modal logic, as opposed to multi-agent epistemic logic.

% The present work
The present work extends the future event logic to the setting for which it was
originally intended: epistemic logic. We will refer to the future event logic
using the more general name of {\em refinement quantified modal logics}. We
consider refinement quantified versions of modal, doxastic and epistemic logics.
We provide axiomatisations for the refinement quantified epistemic and doxastic
logics in the single-agent case, and for the refinement quantified modal and
doxastic logics in the multi-agent case. As our axiomatisations involve
translating the refinement quantified logics into basic modal logics, this gives
us decidability results, and upper bounds on the complexities of these logics,
amongst other results.
