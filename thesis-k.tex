In this chapter we will introduce the semantics of the refinement quantified
modal logics, as introduced by van Ditmarsch and
French~\cite{french2009simulation}. van Ditmarsch, French and
Pinchinat~\cite{french2010future} later provided an axiomatisation for the
single-agent refinement quantified modal logic, in the class of \classK{}. We
extend these results with an axiomatisation for the multi-agent case.

\section{Syntax and semantics}

We begin with a definition of the language, \langF{}, of the refinement
quantified modal logic, along with a definition of its semantics over a general
class of frames.

\begin{definition}[Language of \langF{}]
Given a finite set of agents $A$ and a set of propositional atoms $P$, the
language of \langF{} is defined by the following abstract syntax:
$$
\phi ::=    p \bnfalt
            \neg \phi \bnfalt
            \phi \land \phi \bnfalt
            \knows_a \phi \bnfalt
            \allrefs_a \phi
$$
where $a \in A$ and $p \in P$.
\end{definition}

We use the standard abbreviations from basic modal logic, along with an
abbreviation for the dual of the $\allrefs_a$ operator, $\somerefs_a \phi ::=
\neg \allrefs_a \neg \phi$.

We also use the cover operator $\covers_a \Gamma$, where $\Gamma$ is a finite
set of formulae, which is an abbreviation for 
$\covers_a \Gamma ::= \knows_a \bigvee_{\gamma \in \Gamma} \gamma \land
\bigwedge_{\gamma \in \Gamma} \suspects_a \gamma$. The cover operator is relied
on for our axiomatisation, in much the same way it is relied on for the
axiomatisation of \logicKiF{} presented by van Ditmarsch, French and
Pinchinat~\cite{french2010future}. % TODO - include citation for cover operator

\begin{definition}[Semantics of \logicCF{}]
Let \classC{} be a class of Kripke models, and let $M = (S, R, V) \in \classC$
be a Kripke model taken from \classC{}. The interpretation of $\phi$ is defined
inductively.
\begin{eqnarray*}
M_s &\entails& p \text{ iff } s \in V_p\\
M_s &\entails& \neg \phi \text{ iff } M_s \nentails \phi\\
M_s &\entails& \phi \land \psi \text{ iff } M_s \entails \phi \text{ and } M_s
\entails \psi\\
M_s &\entails& \knows_a \phi \text{ if for all } t \in S : (s, t) \in R_a \text{
implies } M_t \entails \phi\\
M_s &\entails& \allrefs_a \phi \text{ iff for all } M'_{s'} \in \classC : M_s
\simulation_a M'_{s'} \text{ implies } M'_{s'} \entails \phi\\
\end{eqnarray*}
\end{definition}

We use the same terminology and notation for satisfiability and validity as is
used for basic modal logics. As we sometimes discuss basic modal logics and
refinement quantified modal logics at the same time, we may add a subscript to
the turnstile operator, e.g. $\classK \entails_\somerefs \phi$, to make it
explicit that we are working in the refinement quantified version of the logic.

The logics \logicKF{}, \logicKDF{} and \logicSF{} are instances of \logicCF{},
with classes \classK{}, \classKD{} and \classS{} respectively. 

It should be emphasised that the interpretation of the refinement quantifier,
$\allrefs_a$, varies for each logic, as the refinements considered in the
interpretation of each logic must be taken from the appropriate class of Kripke
models. It is for this reason that \logicSF{} and \logicKDF{} are not
conservative extensions of \logicKF{}. For example, $\somerefs_a \knows_a \bot$
is valid in \logicKF{}, but not in \logicSF{} or \logicKDF{}. This is because
given any pointed model in \classK{}, one can construct an $a$-refinement from
that model by deleting the $a$-edges starting at the designated state; in this
resulting $a$-refinement, $\knows_a \bot$ is satisfied, and hence $\somerefs_a
\knows_a \bot$ is satisfied in the original model. However because of the serial
property of \classS{} and \classKD{} models, $\knows_a \bot$ is not even
satisfiable in \logicSF{} or \logicKDF{}, and hence $\somerefs_a \knows_a \bot$
is not satisfiable as well.

We also note that for a class of frames \classC{}, the refinement quantified
modal logic \logicCF{} is a conservative extension of the basic modal logic
\logicC{}. This is because the interpretation of any formula that does not
contain a $\allrefs_a$ operator is defined to be the same as the interpretation
of that formula in basic modal logic.

\begin{lemma}
The logic \logicKF{} is bisimulation invariant.
\end{lemma}

This is proven by van Ditmarsch, French and Pinchinat~\cite{french2010future}.

\begin{lemma}
The logics \logicKDF{} and \logicSF{} are bisimulation invariant.
\end{lemma}

The proof for bisimulation invariance in \logicKF{}, given by van Ditmarsch,
French and Pinchinat~\cite{french2010future} applies to \logicSF{} and
\logicKDF{}.

% TODO - examples

\section{Technical preliminaries}

In previous work, van Ditmarsch, French and Pinchinat~\cite{french2010future}
gave an axiomatisation of the single-agent variant of the refinement quantified
modal logic \logicKF{}.  The axiomatisation was formulated in terms of the cover
operator, $\covers$, an abbreviation which we defined previously. The
completeness proof consisted of a provably correct translation of formulae from
refinement quantified modal logic to basic modal logic, a translation which
relied on a disjunctive normal form, using the cover operator. Our
axiomatisation for the multi-agent refinement quantified modal logic relies on
the same disjunctive normal form, which we will define now.

\begin{definition}[Disjunctive normal form]
A formula in disjunctive normal form is defined by the following abstract syntax:
$$
\alpha ::= \pi \land \bigwedge_{a \in B \subseteq A} \covers_a \Gamma_a \bnfalt \alpha \lor \alpha
$$
where $\pi$ stands for a propositional formula, and for $a \in B \subseteq A$,
$\Gamma_a$ stands for a finite set of formulae in disjunctive normal form.
\end{definition}

To show that every \lang{} formula is equivalent to a disjunctive normal
formula, we first introduce the negation normal form and a corresponding lemma
for that form.

\begin{definition}[Negation normal form]
A formula in negation normal form is defined by the following abstract syntax:
$$
\alpha ::= p \bnfalt 
\neg p \bnfalt
\alpha \land \alpha \bnfalt
\alpha \lor \alpha \bnfalt
\knows_a \alpha \bnfalt
\suspects_a \alpha
$$
where $p \in P$ and $a \in A$.
\end{definition}

\begin{lemma}\label{k-nnf}
Every formula of \lang{} is equivalent to a formula in negation normal form,
under the semantics of \logicK{}.
\end{lemma}

\begin{proof}
Similar to negation normal forms in propositional logic, we can recursively push
the negations inwards using the following equivalences:
\begin{eqnarray*}
\neg \neg \phi &\iff& \phi\\
\neg (\phi \land \psi) &\iff& \neg \phi \lor \neg \psi\\
\neg \knows_a \phi &\iff& \suspects_a \neg \phi
\end{eqnarray*}
\end{proof}

\begin{lemma}\label{k-dnf}
Every formula of \lang{} is equivalent to a formula in disjunctive normal form,
under the semantics of \logicK{}.
\end{lemma}

\begin{proof}
Let $\alpha \in \lang$. Without loss of generality, by Lemma~\ref{k-nnf}, we may
assume that $\alpha$ is in negation normal form. We prove by induction over the
structure of $\alpha$ that $\alpha$ is equivalent to a formula in disjunctive
normal form. The induction hypothesis is that all subformulae of $\alpha$ have
equivalents in disjunctive normal form.

The base case is when $\alpha = p$ or $\alpha = \neg p$ for some $p \in P$, in
which case we are done.

Suppose that $\alpha = \phi \lor \psi$. By the induction hypothesis, there are
formulae $\phi'$ and $\psi'$ in disjunctive normal form that are equivalent to
$\phi$ and $\psi$ respectively. Then $\phi \lor \psi \iff \phi' \lor \psi'$,
which is in disjunctive normal form.

Suppose that $\alpha = \knows_a \phi$. By the induction hypothesis, there is a
formula $\phi'$ in disjunctive normal form that is equivalent to $\phi$. Then
$\knows_a \phi \iff \covers_a \{\phi\} \lor \covers_a \emptyset$, which is in
disjunctive normal form.

Suppose that $\alpha = \suspects_a \phi$. By the induction hypothesis, there is
a formula $\phi'$ in disjunctive normal form that is equivalent to $\phi$. Then
$\suspects_a \phi \iff \covers_a \{\phi, \top\}$, which is in disjunctive normal
form.

Suppose that $\alpha = \phi \land \psi$. By the induction hypothesis, there are
formulae $\phi'$ and $\psi'$ in disjunctive normal form that are equivalent to
$\phi$ and $\psi$ respectively. Then $\phi \land \psi \iff \phi' \land \psi'$.
As $\phi'$ and $\psi'$ are in disjunctive normal form, then $\phi' = \delta_1
\lor \cdots \lor \delta_m$ and $\psi' = \gamma_1 \lor \cdots \lor \gamma_m$ for
some $m, n \geq 0$, where each of the $\delta_i$ and $\gamma_i$ are terms of the
form $\pi \land \bigwedge_{a \in B \subseteq A} \covers_a \Gamma_a$.  Then we
can rewrite $\alpha$ as a disjunction of conjunctions, by the following
equivalence:
$$
\phi' \land \psi' \iff \bigvee_{i \leq m, j \leq n} \delta_i \land \gamma_j
$$

For each $i \leq m$ and $j \leq n$, we have that $\delta_i = \pi \land
\bigwedge_{a \in B \subseteq A} \covers_a \Gamma_a$, and $\gamma_j = \rho \land
\bigwedge_{a \in C \subseteq A} \covers_a \Gamma'_a$, where $\pi$ and $\rho$ are
propositional formulae, and each $\Gamma_a$ and $\Gamma'_a$ is a set of
disjunctive normal formulae. Then we can write each conjunction as: 
$$\delta_i \land \gamma_i \iff (\pi \land \rho) \land \bigwedge_{a \in B
\subseteq A} \covers_a \Gamma_a \land \bigwedge_{a \in C \subseteq A} \covers_a
\Gamma'_a$$

We note that the sets of agents $B$ and $C$ may intersect, and hence the same
agent may appear in each of those sets, possibly with different sets of formulae
$\Gamma_a$ and $\Gamma'_a$. We can combine the two sets of formulae into one, so
that each agent appears only once, using the following equivalence:
$$
\covers_a \Gamma \land \covers_a \Gamma' \equiv 
\covers_a \big( 
\{ \gamma \land \bigvee_{\gamma' \in \Gamma'} \gamma' \mid \gamma \in \Gamma \}
\cup
\{ \gamma' \land \bigvee_{\gamma \in \Gamma} \gamma \mid \gamma' \in \Gamma' \}
\big)
$$
We note that as each $\gamma \in \Gamma$ and $\gamma' \in \Gamma'$ are assumed
to be disjunctive normal formulae, that applying a disjunction over each of
these sets yields a disjunctive normal formula. Conjoining two disjunctive
normal formulae does not yield a disjunctive normal formula, however an
inductive argument can be used to show that recursively applying the same
translation described here, to each of these conjunctions, yields a disjunctive
normal formula.

Repeating this for each disjunct in our original formula leaves us with a
formula in cover logic disjunctive normal form.
\end{proof}

We note that, similar to disjunctive normal forms in propositional logic, the
translation into disjunctive normal form in modal logic results in a formula
that is exponentially larger than the original formula in the worst case.

\section{Axiomatisation}

We provide an axiomatisation of the multi-agent refinement quantified modal
logic, \logicKF{}, and prove its soundness and completeness.

\begin{definition}[Axiomatisation \axiomKF]
The axiomatisation \axiomKF{} is a substitution schema consisting of the
following axioms:
$$
\begin{array}{rl}
{\bf P} & \text{All propositional tautologies}\\
{\bf K} & \knows (\phi \implies \psi) \implies \knows \phi \implies \knows
\psi\\
{\bf R} & \allrefs_a (\phi \implies \psi) \implies \allrefs_a \phi \implies
\allrefs_a \psi\\
{\bf RP} & \allrefs_a \alpha \iff \alpha \text{ where $\alpha$ is a
propositional formula}\\
{\bf RComm} & \somerefs_a \covers_b \Gamma \iff \covers_b \{\somerefs_a \gamma
\mid \gamma \in \Gamma\} \text{ where $a \neq b$}\\
{\bf RDist} & \bigwedge_{b \in A} \somerefs_a \covers_b \Gamma_b \implies
\somerefs_a \bigwedge_{b \in A} \covers_b \Gamma_b\\
{\bf RK} & \somerefs_a \covers_a \Gamma \iff \bigwedge_{\gamma \in \Gamma}
\suspects_a \somerefs_a \gamma\\
\end{array}
$$
Along with the rules:
$$
\begin{array}{rl}
{\bf MP} & \text{From $\proves \phi \implies \psi$ and $\proves \phi$, infer
$\proves \psi$}\\
{\bf NecK} & \text{From $\proves \phi$ infer $\proves \knows_a \phi$}\\
{\bf NecR} & \text{From $\proves \phi$ infer $\proves \allrefs_a \phi$}
\end{array}
$$
\end{definition}

% TODO - compare to single-agent axiomatisation
The axiomatisation \axiomKF{} shares many of the axioms and rules of the
axiomatisation from the single-agent case. The axioms {\bf P}, {\bf K}, {\bf R},
{\bf RP} and {\bf RK}, and the rules {\bf MP}, {\bf NecK} and {\bf NecR} are
essentially the same as the axioms that van Ditmarsch, French and
Pinchinat~\cite{french2010future} used in the single-agent case. The differences
are that \axiomKF{} contains axioms for handling the interaction between
multiple agents. The axioms {\bf RComm} and {\bf RDist} are novel axioms used to
handle the situation where a refinement quantifier is applied to a cover
operator of a different agent, and where a refinement quantifier is applied to a
conjunction of cover operators belonging to different agents.

We will now show that the axiomatisation is sound with respect to \classK{}
models.

\begin{lemma}\label{k-sound}
The axiomatisation \axiomKF{} is sound in \logicKF{}.
\end{lemma}

\begin{proof}
The soundness of the axioms {\bf P}, and {\bf K}, and the rules {\bf MP} and
{\bf NecK} can be shown by the same reasoning used to show that they are sound
in \logicK{}. As the axioms {\bf RP} and {\bf R}, and the rule {\bf NecR}
involve only a single agent, their soundness can be shown by the same reasoning
used to how that they are sound in the single-agent refinement quantified modal
logic~\cite{french2010future}.

All that remains to be shown is the soundness of {\bf RK}, {\bf RComm}, and {\bf
RDist}.

\paragraph{RK}
Suppose that $M_s \in \classK$ is a Kripke model such that $M_s \entails
\bigwedge_{\gamma \in \Gamma} \suspects_a \somerefs_a \gamma$.

We need to show that $M_s \entails \somerefs_a \covers_a \Gamma$. To do this we
will construct a model, $N_t$, from parts of $M_s$, construct an $a$-simulation
from $N_t$ to $M_s$ to show that $N_t \refinement_a M_s$, and finally show that
$N_t \entails \covers_a \Gamma$.

We begin by constructing the model $N_t$. Consider $\gamma \in \Gamma$. From
$M_s \entails \suspects_a \somerefs_a \gamma$, there exists a state $s^\gamma
\in sR^M$ such that $M_{s^\gamma} \entails \somerefs_a \gamma$. Therefore there
exists a Kripke model $N^\gamma_{t^\gamma} \refinement_a M_{s^\gamma}$, via some
$a$-simulation $\mathcal{R}^\gamma$, such that $N^\gamma_{t^\gamma} \entails
\gamma$. Without loss of generality we assume that each $N^\gamma$ is disjoint.

Let $t$ be a state not in $S^M$ or any of the $N^\gamma$. Then we construct a
Kripke model $N = (S^N, R^N, V^N)$ where:
\begin{eqnarray*}
S^N &=& \{t\} \cup S^M \cup \bigcup_{\gamma \in \Gamma} S^{N^\gamma}\\
R^N_a &=& \{(t, t^\gamma) \mid \gamma \in \Gamma\}
\cup R^M_a
\cup \bigcup_{\gamma \in \Gamma} R^{N^\gamma}_a\\
R^N_b &=& \{(t, t') \mid t' \in sR^M_b\}
\cup R^M_b
\cup \bigcup_{\gamma \in \Gamma} R^{N^\gamma}_b \text{ for $b \in A - \{a\}$}\\
V^N(p) &=& 
\begin{cases}
\displaystyle \{t\} \cup V^M(p) \cup \bigcup_{\gamma \in \Gamma} V^{N^\gamma}(p) & \text{if $s
\in V^M(p)$}\\
\displaystyle V^M(p) \cup \bigcup_{\gamma \in \Gamma} V^{N^\gamma}(p) & \text{otherwise}
\end{cases}
\text{ for $p \in P$}
\end{eqnarray*}

\begin{figure}
\begin{center} % TODO - better diagram
\scalebox{0.4}{
\includegraphics{rk}
}
\caption{
The model $N$ is constructed by taking the model $M$ and the models $N^\gamma$
for every $\gamma \in \Gamma$, and connecting them with an extra node $t$. $t$
is connected via an $a$-edge to $t^\gamma$ from each of the $N^\gamma$, and is
also connected via a $b$-edge to each $b$-successor of $s$ in $M$.
}
\end{center}
\end{figure}

We construct an $a$-simulation $\mathcal{R}$ from $N_t$ to $M_s$, where:
$$\mathcal{R} = \{(t, s)\} \cup \{(s', s') \mid s' \in S^M \} 
\cup \bigcup_{\gamma \in \Gamma} \mathcal{R}^\gamma$$

We must show that $\mathcal{R}$ satisfies {\bf atoms}, {\bf forth-$b$} for every
$b \in A$, and {\bf back-$b$} for every $b \in A - \{a\}$.

\paragraph{atoms} We note that, by construction, the valuation of $N$ matches
the valuation of its corresponding states in $M$ and each $N^\gamma$, and the
valuation of $N_t$ matches that of $M_s$. Therefore $\mathcal{R}$ satisfies {\bf
atoms}.

\paragraph{forth} We next show that $\mathcal{R}$ satisfies {\bf forth-$b$}
for every $b \in A$.  Let $b \in A$ and let $(u, v) \in \mathcal{R}$.

Suppose that $(u, v) \in \mathcal{R}^\gamma$ for some $\gamma \in \Gamma$.  Then
as $\mathcal{R}^\gamma$ is an $a$-simulation, it satisfies {\bf forth-$b$} for
every $b \in A$. Hence for every $u' \in uR^{N^\gamma}_b = uR^N_b$, there exists
some $v' \in vR^M_b$ such that $(u', v') \in \mathcal{R}^\gamma \subseteq
\mathcal{R}$. 

Suppose instead that $(u, v) = (s', s')$ for some $s' \in S^M$.  Then we note
that $s'R^N_b = s'R^M_b$, and hence for every $s'' \in s'R^N_b$ we have that
$s'' \in s'R^M_b$, and that $(s'', s'') \in \mathcal{R}$. 

Finally suppose that $(u, u') = (t, s)$. We must consider the cases where $b =
a$ and where $b \neq a$. So suppose that $b = a$. By construction, $tR^N_a =
\{t^\gamma \mid \gamma \in \Gamma\}$, and hence $v = t^\gamma$ for some $\gamma
\in \Gamma$. Hence we can take $s^\gamma \in sR^M_a$, and note that as
$\mathcal{R}^\gamma$ is an $a$-simulation from $M_{s^\gamma}$ to
$N^\gamma_{t^\gamma}$, we know that $(t^\gamma, s^\gamma) \in \mathcal{R}^\gamma
\subseteq \mathcal{R}$. Suppose that $b \neq a$. Then by construction, $tR^M_b =
sR^M_b$, hence for every $t' \in tR^M_b$, we have that $t' \in sR^M_b$, and
hence we know that $(t', t') \in \mathcal{R}$. 

Therefore $\mathcal{R}$ satisfies {\bf forth-$b$} for every $b \in A$.

\paragraph{back} A similar argument to above shows that $\mathcal{R}$
satisfies {\bf back-$a$} for every $b \in A - \{a\}$.

Therefore $\mathcal{R}$ is an $a$-simulation, and $N_t \refinement_a M_s$.

Finally we show that $N_t \entails \covers_a \Gamma$. We must show that for each
$\gamma \in \Gamma$ that $N_{t^\gamma} \entails \gamma$. This follows from the
fact that $N_{t^\gamma}$ is bisimilar to $N^\gamma_{t^\gamma}$. This is obvious,
as $N$ contains a duplicate of $N^\gamma$, and $N$ does not add any additional
edges originating from states in $S^{N^\gamma}$. Hence from bisimulation
invariance, $N_{t^\gamma} \entails \gamma$ for every $\gamma \in \Gamma$, and
hence $N_t \entails \covers_a \Gamma$.

As $N_t \refinement_a M_s$, and $N_t \entails \covers_a \Gamma$ we therefore
have that $M_s \entails \somerefs_a \covers_a \Gamma$.

Conversely, suppose that $M_s \entails \covers_a \Gamma$. Then there exists a
Kripke model $N_t \refinement_a M_s$, via some $a$-simulation $\mathcal{R}$,
such that $N_t \entails \covers_a \Gamma$. From the definition of the cover
operator, this implies that $N_t \entails \knows_a \bigvee_{\gamma \in \Gamma}
\gamma \land \bigwedge_{\gamma \in \Gamma} \suspects_a \gamma$. In particular we
note that for every $\gamma \in \Gamma$, $N_t \entails \suspects_a \gamma$, and
so there exists some $t^\gamma \in tR^N_a$ such that $N_{t^\gamma} \entails
\gamma$. As $t^\gamma \in tR^N_a$, and $(t, s) \in \mathcal{R}$, by {\bf
forth-$a$} there exists some $s^\gamma \in sR^M_a$ such that $(t^\gamma, s^\gamma)
\in \mathcal{R}$. Hence $\mathcal{R}$ is also an $a$-simulation from
$N_{t^\gamma}$ to $M_{s^\gamma}$, and so $M_{s^\gamma} \entails \somerefs_a
\gamma$. As for every $\gamma \in \Gamma$ we have that $s^\gamma \in sR^M_a$, we
also have that $M_s \entails \suspects_a \somerefs_a \gamma$. Therefore we
finally have that $M_s \entails \bigwedge_{\gamma \in \Gamma} \suspects_a
\somerefs_a \gamma$.

Therefore {\bf RK} is sound.

\paragraph{RComm}
Suppose that $M_s \in \classK$ is a Kripke model such that $M_s \entails
\covers_b \{ \somerefs_a \gamma \mid \gamma \in \Gamma\}$, where $a \neq b$.

We need to show that $M_s \entails \somerefs_a \covers_b \Gamma$. To do this we
follow the same strategy as for proving {\bf RK}: we construct an $a$-refinement
$N_t$, and show that $N_t \entails \covers_b \Gamma$.

We begin by constructing the model $N_t$. Consider $\gamma \in \Gamma$. From
$M_s \entails \covers_b \{ \somerefs_a \gamma \mid \gamma \in \Gamma \}$, there
exists a state $s^\gamma \in sR^M_b$ such that $M_{s^\gamma} \entails
\somerefs_a \gamma$. Therefore there exists a Kripke model $N^\gamma_{t^\gamma}
\refinement_a M_{s^\gamma}$, via some $a$-simulation $\mathcal{R}^\gamma$, such
that $N^\gamma_{t^\gamma} \entails \gamma$. Without loss of generality we assume
that each $N^\gamma$ is disjoint.

Let $t$ be a state not in $S^M$ or any of the $N^\gamma$. Then we construct a
Kripke model $N = (S^N, R^N, V^N)$ where:
\begin{eqnarray*}
S^N &=& \{t\} \cup S^M \bigcup_{\gamma \in \Gamma} S^{N^\gamma}\\
R^N_b &=& \{(t, t^\gamma) \mid \gamma \in \Gamma\} 
\cup  R^M_b 
\cup \bigcup_{\gamma \in \Gamma} R^{N^\gamma}_b\\
R^N_c &=& \{(t, t') \mid t' \in sR^M_c\} 
\cup R^M_c \cup \bigcup_{\gamma \in \Gamma} R^{N^\gamma}_c \text{ for $c \in A - \{b\}$}\\
V^N(p) &=& 
\begin{cases}
\displaystyle \{t\} \cup V^M(p) \cup \bigcup_{\gamma \in \Gamma} V^{N^\gamma}(p)
& \text{if $s \in V^M(p)$}\\
\displaystyle V^M(p) \cup \bigcup_{\gamma \in \Gamma} V^{N^\gamma}(p) &
\text{otherwise}
\end{cases}
\text{ for $p \in P$}
\end{eqnarray*}

We construct an $a$-simulation $\mathcal{R}$ from $N_t$ to $M_s$, where:
$$\mathcal{R} = \{(t, s)\} \cup \{(s', s') \mid s' \in S^M\} \cup
\bigcup_{\gamma \in \Gamma} \mathcal{R}^\gamma$$

We note that $\mathcal{R}$ is an $a$-simulation, by similar arguments as used in
the proof for {\bf RK}. In particular, this means that $N_t \refinement_a M_s$.

We also note that for every $\gamma \in \Gamma$ that $N_{t^\gamma} \bisim
N^\gamma_{t^\gamma}$, by similar arguments as used in the proof for {\bf RK}.
In particular, this means that as $N^\gamma_{t^\gamma} \entails \gamma$ that we
also have $N_{t^\gamma} \entails \gamma$, for every $\gamma \in \Gamma$.
Therefore $N_t \entails \covers_b \Gamma$.

Therefore $M_s \entails \somerefs_a \covers_b \Gamma$.

The converse, $\somerefs_a \covers_b \Gamma \implies \covers_b \{\somerefs_a
\gamma \mid \gamma \in \Gamma\}$ follows a similar proof to the relevant part in
the proof for {\bf RK}.

Therefore {\bf RComm} is sound.

\paragraph{RDist}
Suppose that $M_s \in \classK$ is a Kripke model such that $M_s \entails
\bigwedge_{b \in A} \somerefs_a \covers_b \Gamma_b$.

We need to show that $M_s \entails \somerefs_a \bigwedge_{b \in A} \covers_b
\Gamma_b$. To do this we follow the same strategy as for proving {\bf RK}: we
construct an $a$-refinement $N_t$, and show that $N_t \entails \somerefs_a
\bigwedge_{b \in A} \covers_b \Gamma_b$.

We begin by constructing the model $N_t$. As $a \in A$, we have that $M_s
\entails \somerefs_a \covers_a \Gamma_a$, and by {\bf RK} this implies that $M_s
\entails \bigwedge_{\gamma \in \Gamma_a} \suspects_a \somerefs_a \gamma$. We
also have for every $b \in A - \{a\}$ that $M_s \entails \somerefs_a \covers_b
\Gamma_b$, and by {\bf RComm} this implies that $M_s \entails \covers_b
\{\somerefs_a \gamma \mid \gamma \in \Gamma_b\}$, and by the definition of the
cover operator implies that $M_s \entails \bigwedge_{\gamma \in \Gamma_b}
\suspects_b \somerefs_a \gamma$.  Hence for every $b \in A$ and $\gamma \in
\Gamma_b$, we have that $\suspects_b \somerefs_a \gamma$. This implies that for
each $b \in A$ and each $\gamma_b \in \Gamma_b$ that there exists some $t^{\gamma}
\in sR^M_b$ such that $M_{t^\gamma} \entails \somerefs_a \gamma$. Therefore
there exists a Kripke model $N^\gamma_{t^\gamma} \refinement_a M_{t^\gamma}$,
via some $a$-simulation relation $\mathcal{R}^\gamma$, such that
$N^\gamma_{t^\gamma} \entails \gamma$.  Without loss of generality we may assume
that the $N^\gamma$ are disjoint.

Let $t$ be a state not in $S^M$ or any of the $N^\gamma$. Then we construct a
Kripke model $N = (S^N, R^N, V^N)$ where:

\begin{eqnarray*}
S^N &=& \{t\} \cup \bigcup_{b \in A, \gamma \in \Gamma_b} S^{N^\gamma}\\
R^N_b &=& \{(t, t^{\gamma}) \mid b \in A, \gamma \in \Gamma_b\} 
\cup \bigcup_{b \in A, \gamma \in \Gamma_b} R^{N^\gamma}_b \text{ for $b \in A$}\\
V^N(p) &=& 
\begin{cases}
\{t\} \cup \bigcup_{b \in A, \gamma \in \Gamma_b} V^{N^\gamma}(p) & \text{if $s
\in V^M(p)$}\\
\bigcup_{b \in A, \gamma \in \Gamma_b} V^{N^\gamma}(p) & \text{otherwise}
\end{cases}
\end{eqnarray*}

We construct an $a$-simulation $\mathcal{R}$ from $N_t$ to $M_s$, where:
$$\mathcal{R} = \{(t, s)\} \cup \bigcup_{b \in A, \gamma \in \Gamma_b}
\mathcal{R}^\gamma$$

We note that this is an $a$-simulation, by similar arguments as used in the
proof for {\bf RK}. In particular, this means that $N_t \refinement_a M_s$.

We also note that for every $b \in A$, and $\gamma \in \Gamma_b$ that
$N_{t^\gamma} \bisim N^\gamma_{t^\gamma}$, by similar arguments as used in the
proof for {\bf RK}. In particular, this means that as $N^\gamma_{t^\gamma}
\entails \gamma$ that we also have $N_{t^\gamma} \entails \gamma$, for every $b
\in A$ and $\gamma \in \Gamma_b$. Therefore $N_t \entails \covers_b \Gamma_b$
for every $b \in A$, and therefore $N_t \entails \bigwedge_{b \in A} \covers_b
\Gamma_b$.

Therefore $M_s \entails \somerefs_a \bigwedge_{b \in A} \covers_b \Gamma_b$ and
{\bf RDist} is sound.

Therefore the axiomatisation \axiomKF{} is sound.
\end{proof}

We note that the implication in {\bf RDist} can actually be strengthened to an
equality, but that this is derivable from the other axioms in \axiomKF{}.

\begin{lemma}
The following is derivable in \axiomKF{}.
$$
\proves \bigwedge_{b \in A} \somerefs_a \covers_b \Gamma_b \iff
\somerefs_a \bigwedge_{b \in A} \covers_b \Gamma_b \\
$$
where $\Gamma_b$ is a set of $b$-disjunctive normal formulae for
every $b \in A$.
\end{lemma}

\begin{proof}
The forward direction is the axiom {\bf RDist}. 

The converse can be derived in a more general form as $\somerefs_a (\phi \land
\psi) \implies \somerefs_a \phi \land \somerefs_a \psi$. The derivation is
similar to the derivation for $\knows_a (\phi \land \psi) \implies \knows_a \phi
\land \knows_a \psi$ in the modal logic \logicK{}, using the axiom {\bf R} in
place of {\bf K}.
\end{proof}

We show the completeness of the axiomatisation \axiomKF{} by a provably correct
translation from \langF{} to \lang{}. Completeness then follows from the
completeness of \logicK{}.

\begin{lemma}\label{k-translation}
Every formula of \langF{} is provably equivalent to a formula of \lang{} with
the axiomatisation \axiomKF{}.
\end{lemma}

\begin{proof}
Given a formula $\psi$ we prove by induction on the number of occurrences of
\somerefs{} that $\psi$ is equivalent to a \somerefs-free formula, and
therefore to a formula in \lang{}. The base case with no \somerefs operators
is trivial, as a \somerefs-free formula is a formula in \lang{}. Now assume
that $\psi$ contains $n + 1$ \somerefs-operators. Choose a subformula of type
$\somerefs_a \phi$ of our given formula, where $\phi$ is \somerefs-free. Without
loss of generality, by Lemma~\ref{k-dnf} we may assume that $\phi$ is in
disjunctive normal form.  We prove by induction on the structure of $\phi$ that
$\somerefs_a \phi$ is provably equivalent to a formula $\chi$ without the
$\somerefs_a$ operator.

\begin{itemize}
\item $\somerefs_a (\phi \lor \psi)$ iff $\somerefs \phi \lor \somerefs \psi$.
(Derivable from {\bf P} and {\bf R})
\item $\somerefs_a (\pi \land \bigwedge_{b \in A} \covers_b \Gamma_b)$ iff
$\pi \land \bigwedge_{\gamma \in \Gamma_a} \suspects_a \somerefs_a \gamma \land
\bigwedge_{b \in A - \{a\}} \covers_b \{\somerefs_a \gamma \mid \gamma \in
\Gamma_b\}$ (Derivable from {\bf P}, {\bf R}, {\bf RP}, {\bf RDist}, {\bf
RK} and {\bf RComm})
\end{itemize}

Replacing $\somerefs_a \phi$ with $\chi$ in $\psi$ gives an equivalent formula
with one less \somerefs-operator. Thus by induction, all formulae in \langF{}
can be translated into an equivalent formula in \lang{} using the axiomatisation
\axiomKF{}.
\end{proof}

The rest of the completeness proof is merely a formality to show that, given the
above translation into \lang{}, we can show completeness by using these
translations along with the completeness of \logicK{}.

\begin{corollary}\label{k-derivable}
Let $\phi \in \langF$ be given and $\psi \in \lang$ be semantically
equivalent to $\phi$.  If $\psi$ is a theorem in \logicK{}, then $\phi$ is a
theorem in \axiomKF{}.
\end{corollary}

\begin{proof}
Let $\phi \in \langF$ and let $\psi \in \lang$ be semantically equivalent to
$\phi$. By Lemma~\ref{k-translation}, we can obtain some $\phi' \in \lang$
that is semantically equivalent to $\phi$ (and thus also to $\psi$) by following
the given translation steps. We can extend a derivation of $\psi$ to a
derivation of $\phi'$ as the two are semantically equivalent in \logicK{}, and by
the completeness of \logicK{} this equivalence is derivable. As \axiomKF{} is a
conservative extension of \logicK{}, this equivalence is therefore also derivable
in \axiomKF{}. The derivation can be further extended to $\phi$ by observing that all
of the reduction steps in Lemma~\ref{k-translation} are provable equivalences
in \axiomKF{}. Therefore $\phi$ is a theorem in \axiomKF{}.
\end{proof}

\begin{lemma}\label{k-complete}
The axiom schema \axiomKF{} is complete for the logic \logicKF{}.
\end{lemma}

\begin{proof}
Let $\phi \in \langF$ such that $\classK \entails_\somerefs \phi$. Then by
Lemma~\ref{k-translation}, there exists a semantically equivalent formula
$\psi \in \lang$ which is \somerefs-free. As $\classK \entails_\somerefs \phi$ and
$\phi \iff \psi$, then $\classK \entails_\somerefs \psi$. As $\psi$ is
\somerefs-free, then it follows that $\classK \entails \psi$, and by the
completeness of \axiomKF{} it follows that $\proves_{\axiomK} \psi$.
Therefore by Corollary~\ref{k-derivable} we have that $\proves_{\axiomKF}
\phi$.
\end{proof}

\begin{theorem}
The axiomatisation \axiomKF{} is sound and complete for the logic \logicKF{}.
\end{theorem}

\begin{proof}
The soundness proof is given in Lemma~\ref{k-sound} and the completeness
proof is given in Lemma~\ref{k-complete}.
\end{proof}
