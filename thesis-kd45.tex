\chapter{Doxastic logic}

\section{Syntax and semantics}

Here we define the syntax and semantics of the logic \logicKDF{}, which
restricts the logic \logicK{}, as defined by van Ditmarsch and
French~\cite{french2009simulation}, to deal with only models and refinements of
models that are in \classKD{}.

\begin{definition}[Language of \langF{}]
Given a finite set of agents $A$ and a set of propositional atoms $P$, the
language of \langF{} is defined by the following abstract syntax:

$$
\phi ::=    p \bnfalt
            \neg \phi \bnfalt
            \phi \land \phi \bnfalt
            \knows_a \phi \bnfalt
            \allrefs_a \phi
$$
where $a \in A$ and $p \in P$.
\end{definition}

Standard abbreviations include:
$\top ::= \phi \lor \neg \phi$;
$\bot ::= \neg \top$;
$\phi \lor \psi ::= \neg (\neg \phi \land \neg \psi)$;
$\phi \implies \psi ::= \neg \phi \lor \psi$;
and $\suspects_a \phi ::= \neg \knows_a \neg \phi$.
We use an abbreviation for the dual of the $\allrefs_a$ operator,
$\somerefs_a \phi ::= \neg \allrefs_a \neg \phi$.

We also use the cover operator $\covers_a \Gamma$, where $\Gamma$ is a finite
set of formulae, which is an abbreviation for 
$\covers_a \Gamma ::= \knows_a \bigvee_{\gamma \in \Gamma} \gamma \land
\bigwedge_{\gamma \in \Gamma} \suspects_a \gamma$. The cover operator is relied
on for our axiomatisation, in much the same way it is relied on for the
axiomatisation of \logicKiF{} presented by van Ditmarsch, French and
Pinchinat~\cite{french2010future}.

\begin{definition}[Semantics of \logicCF]
Let $M = (S, R, V)$ be a doxastic model. The interpretation of $\phi \in
\logicKDF$ is defined inductively.

\begin{eqnarray*}
M_s &\entails& p \text{ iff } s \in V_p\\
M_s &\entails& \neg \phi \text{ iff } M_s \nentails \phi\\
M_s &\entails& \phi \land \psi \text{ iff } M_s \entails \phi \text{ and } M_s
\entails \psi\\
M_s &\entails& \knows_a \phi \text{ if for all } t \in S : (s, t) \in R_a \text{
implies } M_t \entails \phi\\
M_s &\entails& \allrefs_a \phi \text{ iff for all } M'_{s'} \in \classKD : M_s
\simulation_a M'_{s'} \text{ implies } M'_{s'} \entails \phi\\
\end{eqnarray*}
\end{definition}

The difference between \logicKF{} and \logicKDF{} is in the class of models that
they are interpreted over. It should be emphasised that the interpretation of
the refinement operator, $\allrefs_a$, varies for each logic, as the refinements
considered in the interpretation of the operator in \logicKDF{} must be taken
from the class of doxastic models, \classKD{}, whereas in \logicKF{} they may be
arbitrary Kripke models from \classK{}. It is for this reason that \logicKDF{}
is not a conservative extension of \logicKF{}. For example, whilst $\somerefs_a
\knows_a \bot$ is valid in \logicKF{}, as given any Kripke model $M_s$, one can
always take the refinement of $M_s$ such that $s$ has no successors, a model
where $\knows_a \bot$ is satisfied, this formula is not valid in \logicKDF{}, as
every doxastic model has the serial property, and thus $\knows_a \bot$ is not
valid in any doxastic model.

\begin{lemma}
The logic \logicKDF{} is bisimulation invariant.
\end{lemma}

The proof for the bisimulation invariance of \logicKF{}, given by van Ditmarsch,
French and Pinchinat~\cite{french2010future} also applies to \logicKDF{}.

% TODO - examples

In previous work, we gave an axiomatisation of the single-agent logic,
\logicKDiF{}. The soundness and completeness proofs of this axiomatisation
relied on a special normal form for the well-formed formulae of \logicKDi{},
called the prenex normal form. This normal form had the effect of eliminating
nested modal operators ($\knows$ and $\suspects$), and was thus used to avoid the
need to address the transitivity of \classKD{} models when showing soundness and
completeness of our axiomatisations. 

A prenex normal form is not possible for the multi-agent logic of \logicKD{},
however the disjunctive normal form which we are about to define will eliminate
nested modal operators from the same agent, thus a $\knows_a$ operator cannot
appear directly within the scope of another $\knows_a$ or $\suspects_a$
operator. This will achieve the same purpose as the prenex normal form.

\begin{definition}[Disjunctive normal form]
A formula in $a$-disjunctive normal form is defined by the following abstract syntax:

\begin{eqnarray*}
\alpha &::=& \delta \bnfalt \alpha \lor \alpha\\
\delta &::=& \pi \bnfalt \knows_b \gamma_b \bnfalt \suspects_b \gamma_b \bnfalt
\delta \land \delta\\
\end{eqnarray*}

Where $\pi$ stands for a propositional formula, $b \in A - \{a\}$, and
$\gamma_b$ stands for a formula in $b$-disjunctive normal form.

A formula in disjunctive normal form is defined by the following abstract syntax:

\begin{eqnarray*}
\alpha &::=& \delta \bnfalt \alpha \lor \alpha\\
\delta &::=& \pi \bnfalt \knows_a \gamma_a \bnfalt \suspects_a \gamma_a \bnfalt
\delta \land \delta\\
\end{eqnarray*}

Where $\pi$ stands for a propositional formula, $a \in A$, and $\gamma_a$
stands for a formula in $a$-disjunctive normal form.
\end{definition}

\begin{lemma}\label{kd45-dnf-equivalences}
We have the following equivalences in \logicKD{}:

\begin{eqnarray*}
\knows_a (\pi \lor (\alpha \land \knows_a \beta)) &\iff& (\knows_a (\pi \lor \alpha)
\land \knows_a \beta) \lor (\knows_a \pi \land \neg \knows_a \beta)\\
\knows_a (\pi \lor (\alpha \land \suspects_a \beta)) &\iff& (\knows_a (\pi \lor \alpha)
\land \suspects_a \beta) \lor (\knows_a \pi \land \neg \suspects_a \beta)
\end{eqnarray*}
\end{lemma}

This is proven by Meyer and van der Hoek~\cite{meyer2004epistemic} for
\logicSi{}, however the same proof also applies to \logicKD{}.

\begin{lemma}\label{kd45-dnf}
Every well-formed formula of \logicKD{} is equivalent to a formula in
disjunctive normal form.
\end{lemma}

\begin{proof}
We use a proof similar to the proof for prenex normal form, given by Meyer and
van der Hoek~\cite{meyer2004epistemic}.

Let $\phi$ be a well-formed formula of \logicKD{}. We proceed by induction on
the structure of $\phi$, with the induction hypothesis that every strict
subformula of $\phi$ is equivalent to a formula in disjunctive normal form.

Suppose that $\phi$ is a propositional formula. Then we are done; $\phi$ is in
disjunctive normal form.

Suppose that $\phi = \neg \alpha$. Then by the induction hypothesis, $\alpha$ is
equivalent to some $\alpha'$ in disjunctive normal form. We can use an inductive
argument, over $\alpha'$, to show that the negation can be pushed inwards until
the only negations are applied to propositional formulae (the result of which is
a propositional formula), thus yielding a formula equivalent to $\phi$, which is
in disjunctive normal form.

Suppose that $\phi = \alpha \land \beta$. Then by the induction hypothesis, $\alpha$
and $\beta$ are equivalent to some $\alpha'$ and $\beta'$ in disjunctive normal
form. We note that we can equivalently write $\phi$ as $\phi = \neg (\neg
\alpha' \lor \neg \beta')$. From our case for negation, we note that $\neg
\alpha'$ and $\neg \beta'$ are equivalent to some $\alpha''$ and $\beta''$ in
disjunctive normal form. Given these, $\alpha'' \lor \beta''$ is also is
disjunctive normal form. Thus we can use our case for negation once again to
note that $\phi = \neg \alpha'' \lor \beta''$ is equivalent to some $\phi''$ in
disjunctive normal form.

Suppose that $\phi = \knows_a \psi$. Then by the induction hypothesis, $\psi$ is
equivalent to some $\psi'$ in disjunctive normal form. Suppose that $\psi'$ is
not in $a$-disjunctive normal form (otherwise we are done). Then $\psi'$
contains some conjunct of the form $\knows_a \beta$ or $\suspects_a \beta$. Thus
we can rewrite $\psi'$ as $\psi' = \pi \lor (\alpha \land \knows_a \beta)$. By
Lemma \ref{kd45-dnf-equivalences}, we get that $\phi \equiv (\knows_a (\pi \lor
\alpha) \land \knows_a \beta) \lor (\knows_a \pi \land \neg \knows_a \beta)$. We
can use the other equivalence from Lemma \ref{kd45-dnf-equivalences} in the case
that $\phi = \suspects_a \psi$.

Proceeding in this fashion we may move all conjuncts of $\psi'$ containing an
$a$-modality to the outside, so that they are conjuncts of $\phi$, thus
obtaining a formula in disjunctive normal form.
\end{proof}

\begin{definition}[Cover disjunctive normal form]
A formula in $a$-cover disjunctive normal form is defined by the following
abstract syntax:

$$
\alpha ::= \pi \land \bigwedge_{b \in A - \{a\}} \covers_b \Gamma_b \bnfalt
\alpha \lor \alpha
$$

Where $\pi$ stands for a propositional formula, and $\Gamma_b$ stands for a set
of formulae in $b$-cover disjunctive normal form.

A formula in cover disjunctive normal form is defined by the following abstract
syntax:

$$
\alpha ::= \pi \land \bigwedge_{a \in A} \covers_a \Gamma_a \bnfalt
\alpha \lor \alpha
$$

Where $\pi$ stands for a propositional formula, and $\Gamma_a$ stands for a set
of formulae in $a$-cover disjunctive normal form.
\end{definition}

\begin{lemma}
Every well-formed formula of \logicKD{} is equivalent to a formula in
cover disjunctive normal form.
\end{lemma}

\begin{proof}
Without loss of generality, we may assume that our given formula is in
disjunctive normal form (by Lemma \ref{kd45-dnf}).

Given a formula in disjunctive normal form, we consider each disjunct separately. We
can convert each term $\knows_a \gamma$ or $\suspects_a \gamma$ into an equivalent
term using the cover operator, using the equivalences $\knows_a \gamma \equiv
\covers_a \{ \gamma \}$ and $\suspects_a \gamma \equiv \covers_a \{ \gamma, \top
\}$.

An inductive argument can be used to show that we can collapse the resulting
conjunction of cover operators into a single term containing one cover
operator applied to a set of propositional formulae. We use the following
equivalence to achieve this.

$$
\covers_a \Gamma \land \covers_a \Gamma' \equiv 
\covers_a \big( 
\{ \gamma \land \bigvee_{\gamma' \in \Gamma'} \gamma' \mid \gamma \in \Gamma \}
\cup
\{ \gamma' \land \bigvee_{\gamma \in \Gamma} \gamma \mid \gamma' \in \Gamma' \}
\big)
$$

Repeating this for each disjunct in our original formula leaves us with a
formula in cover logic disjunctive normal form.

We note that at each state we never introduce new modalities, rather we convert
$\knows_a$ and $\suspects_a$ operators into $\covers_a$ operators, and then
proceed to collapse $\covers_a$. Thus if a formula is an $a$-disjunctive normal
formula, containing no $a$-modalities, then the result of the above translation
is an $a$-cover disjunctive normal formula.
\end{proof}

The cover logic prenex normal form will be used in our completeness proofs.

\begin{lemma}\label{kd45-successors}
If $\phi$ is a formula in $a$-disjunctive normal form, and $M_s$ is a doxastic
model such that $M_s \entails \phi$, then there exists a model $N_t$ such that
$N_t \entails \phi$ and $tR^N_a = \{t\}$.
\end{lemma}

\begin{proof}
Suppose that $\phi$ is an $a$-disjunctive normal formula, and that $M_s$ is a
doxastic model such that $M_s \entails \phi$. 

Let $t$ be a state such that $t \notin S^{N^\gamma}$ for every $\gamma \in
\Gamma$. Then we construct the model $N = (S^N, R^N, V^N)$, where:

\begin{eqnarray*}
S^N &=& \{t\} \cup S^M\\
R^N_a &=& \{(t, t)\} \cup R^M_a\\
R^N_b &=& \{(t, s') \mid s' \in sR^M_b\} \cup R^M_b \text{ for every $b \in A -
\{a\}$}\\
V^N(p) &=& \begin{cases}
\{t\} \cup V^M(p) & \text{if $s \in V^M(p)$}\\
V^M(p) & \text{otherwise}
\end{cases}
\end{eqnarray*}

First we note that $N$ is a doxastic model. The relation $R^N_a$ consists of the
relation $R^M_a$, combined with the relationship $(t,t)$, which ensures the
serial property given the new element in $S^N$, and preserves transitivity and
Euclideaness, as there is no relationship between $t$ and any other state in
$S^N$. The relation $R^N_b$ for $b \in A - \{a\}$ consists of the relation
$R^M_b$, combined with relationships from $t$ to all of the successors of $s \in
S^M$. As $R^M_b$ is transitive and Euclidean, adding these relationships for $t$
preserves transitivity and Euclideaness. As $R^M_b$ is serial, $sR^M \ne
\emptyset$, and hence $tR^M_b \ne \emptyset$. Therefore $N$ is a doxastic model.

Next we note that for each $b \in A - \{a\}$ and each successor $s' \in sR^M_b$
of $s$, the state $N_{s'}$ is bisimilar to $M_{s'}$, by the bisimulation
relation which maps states in $S^M$ to themselves. This is clear because the
only state in $S^N$ which is not in $S^M$ is $t$, and there are no edges leading
to $t$ from any state in $S^M$.

As $\phi$ is in $a$-disjunctive normal form, $\phi$ has the form $\phi =
\delta_1 \lor \cdots \lor \delta_m$, where each $\delta_i$ has the form
$\delta_i = \gamma_i1 \land \cdots \land \gamma_i{n_i}$, and each $\gamma_{ij}$ is
either a propositional formula, or has the form $\knows_b \psi$ or $\suspects_b
\psi$ for some $b \in A - \{a\}$.  As $M_s \entails \phi$, there exists some $i
= 1, \dots, m$ such that $M_s \entails \delta_i$. Therefore, for each $j = 1,
\dots, n_i$, we have that $M_s \entails \gamma_{ij}$. 

Suppose that $\gamma_{ij}$ is a propositional formula. Then $N_t$ has the same
valuation as $M_s$, and hence is equivalent under propositional formulae.
Therefore $N_t \entails \gamma_{ij}$.  Suppose instead that $\gamma_{ij} = \suspects_b
\psi$ for some $b \in A - \{a\}$ and some formula $\psi$. Then there exists some
$s' \in sR^M_b$ such that $M_{s'} \entails \psi$. From above, we know that
$N_{s'}$ is bisimilar to $M_{s'}$, and so by bisimulation invariance we have
that $N_{s'} \entails \psi$.  By construction, $s' \in t^N_b$, and hence $N_t
\entails \suspects_b \psi$. A similar argument can be used for the case where
$\gamma_{ij} = \knows_b \psi$.  

Hence $N_t \entails \delta_i$ and so $N_t \entails \psi$.
\end{proof}

\section{Axiomatisation}

\begin{definition}[\axiomKDF]
The axiomatisation \axiomKDF{} is a substitution schema consisting of the
following axioms:

$$
\begin{array}{rl}
{\bf P} & \text{All propositional tautologies}\\
{\bf K} & \knows (\phi \implies \psi) \implies \knows \phi \implies \knows
\psi\\
{\bf D} & \knows \phi \implies \suspects \phi\\
{\bf 4} & \knows \phi \implies \knows \knows \phi\\
{\bf 5} & \suspects \phi \implies \knows \suspects \phi\\
{\bf G0} & \allrefs_a (\phi \implies \psi) \implies \allrefs_a \phi \implies
\allrefs_a \psi\\
{\bf G1} & \allrefs_a \alpha \iff \alpha \text{ where $\alpha$ is a
propositional formula}\\
{\bf GKD45} & \somerefs_a \covers_a \Gamma \iff \bigwedge_{\gamma \in \Gamma}
\suspects_a \somerefs_a \gamma \text{ where $\Gamma$ is a set of $a$-disjunctive
normal formulae}\\
{\bf GMulti} & \somerefs_a \covers_b \Gamma \iff \covers_b \{\somerefs_a \gamma
\mid \gamma \in \Gamma\} \text{ where $a \neq b$}\\
{\bf GKD45Multi} & \bigwedge_{b \in A} \somerefs_a \covers_b \Gamma_b \implies
\somerefs_a \bigwedge_{b \in A} \covers_b 
\end{array}
$$

Along with the rules:

$$
\begin{array}{rl}
{\bf MP} & \text{From $\proves \phi \implies \psi$ and $\proves \phi$, infer
$\proves \psi$}\\
{\bf Nec1} & \text{From $\proves \knows_a (\phi \implies \psi)$ and $\proves
\knows_a \phi$, infer $\proves \knows_a \psi$}\\
{\bf Nec2} & \text{From $\proves \allrefs_a (\phi \implies \psi)$ and $\proves
\allrefs_a \phi$, infer $\proves \allrefs_a \psi$}
\end{array}
$$
\end{definition}

\begin{lemma}
The axiomatisation \axiomKDF{} is sound in \logicKDF{}.
\end{lemma}

\begin{proof}
The soundness of the axioms {\bf P}, {\bf K}, {\bf D}, {\bf 4}, and {\bf 5} and the
rules {\bf MP} and {\bf Nec1} can be shown by the same reasoning used to show
that they are sound in \logicKD{}. The soundness of the axioms {\bf G0} and {\bf
G1}, and the rule {\bf Nec2} can be shown by the same reasoning used to show
that they are sound in the \logicKiF{}~\cite{french2010future}.

All that remains to be shown is the soundness of {\bf GKD45}, {\bf GMulti}, and
{\bf GKD45Multi}.

\paragraph{GKD45}
Suppose that $M_s$ is a doxastic model such that $M_s \entails \bigwedge_{\gamma
\in \Gamma} \suspects_a \somerefs_a \gamma$, where $\Gamma$ is a set of
$a$-disjunctive normal formulae.

Then for each $\gamma \in \Gamma$, we have that $M_s \entails \suspects_a
\somerefs_a \gamma$. So there exists a state $s^\gamma \in sR^M$ such
that $M_{s^\gamma} \entails \somerefs_a \gamma$. Therefore there exists a
doxastic model $N_{t^\gamma} \refinement_a M_{s^\gamma}$, via some simulation
relation $\mathcal{R}^\gamma$, such that $N^\gamma_{t^\gamma} \entails \gamma$.
Without loss of generality, we may assume that $t^\gamma R^{N^\gamma} =
\{t^\gamma\}$ (by Lemma \ref{kd45-successors}) and that each of the $N^\gamma$
are disjoint.

Then we can construct a doxastic model $N = (S^N, R^N, V^N)$ where:

\begin{eqnarray*}
S^N &=& \{t\} \cup \bigcup_{\gamma \in \Gamma} S^{N^\gamma}\\
R^N_a &=& \{(t, t^\gamma) \mid \gamma \in \Gamma\} \cup \{(t^\gamma,
t^{\gamma'} \mid \gamma, \gamma' \in \Gamma\} \cup \bigcup_{\gamma \in
\Gamma} R^{N^\gamma}_a\\
R^N_b &=& \{(t, t') \mid \gamma \in \Gamma, s' \in sR^M_b, t' \in S^N, (s', t') \in
\mathcal{R}^\gamma \} \cup \bigcup_{\gamma \in \Gamma} R^{N^\gamma}_b \text{ for
$b \in A - \{a\}$}\\
V^N(p) &=& \begin{cases}
\{t\} \cup \bigcup_{\gamma \in \Gamma} V^{N^\gamma}(p) & \text{if $s \in
V^M(p)$}\\
\bigcup_{\gamma \in \Gamma} V^{N^\gamma}(p) & \text{otherwise}
\end{cases}
\end{eqnarray*}

We note that $N$ is a doxastic model. For each $b \in A - \{a\}$, the relation
$R^N_b$ consists of the union of the relations $R^{N^\gamma}_b$ for each $\gamma
\in \Gamma$, which are each disjoint, and are also serial, transitive and
Euclidean, along with the relationship $\{(t, t') \mid \gamma \in \Gamma, s'
\in sR^M_b, t' \in S^N, (s', t') \in \mathcal{R}^\gamma \}$. The additional
relationship simply relates $t$ to the successors of $s \in S^M$ as they appear
in $N$ (i.e. so they are mapped according to the simulation relations
$\mathcal{R}^\gamma$), and so preserve the transitive and reflexive
properties, and along with the rest of the relations $R^{N^\gamma}_b$, ensures
that $R^N_b$ is serial. The relation $R^N_a$ consists of the union of the
relations $R^{N^\gamma}_a$ for each $\gamma \in \Gamma$, which are each disjoint
and are also serial, transitive and Euclidean, along with the relationships
$\{(t, t^\gamma) \mid \gamma \in \Gamma\} \cup \{(t^\gamma, t^{\gamma'}) \mid
\gamma, \gamma' \in \Gamma\}$. These relationships are not disjoint from the
$R^{N^\gamma}_a$, however as we have assumed that the $t^\gamma R^{N^\gamma}_a =
\{t^\gamma\}$, these are the only parts of these relations that we must consider
to determine if $R^N_a$ is transitive and Euclidean. We observe that the union
of these reflexive relationships, along with the extra part added in $R^N_a$,
almost forms an equivalence class, except for the fact that $(t, t) \notin
R^N_a$.  Hence it $R^N_a$ is transitive and Euclidean. $R^N_a$ is also serial as
each of the $R^{N^\gamma}_a$ are serial for $S^{N^\gamma}$, and $R^N_a$ adds at
least one relation involving $t$.

We note that by construction, for each $\gamma \in \Gamma$, we have that
$N_{t^\gamma} \entails \gamma$. As each $\gamma$ is an $a$-disjunctive normal
formula, a similar argument to that used in Lemma \ref{kd45-successors} can be
used to show that the $a$-edges added to the $t^\gamma$ states does not
effect the truth value of $\gamma$ in $N_{t^\gamma}$, and that in-bound
$b$-edges to $t^\gamma$ do not effect the truth value of any formulae at
$N_{t^\gamma}$. Hence $N_{t^\gamma} \entails \gamma$ for every $\gamma \in
\Gamma$, and for every $t' \in tR^N$, by construction $t' = t^\gamma$ for some
$\gamma \in \Gamma$, and hence $N_{t'} \entails \gamma$. Therefore $N_t \entails
\covers_a \Gamma$.

We next construct an $a$-simulation $\mathcal{R}$ from $N_t$ to $M_s$,
where:

$$\mathcal{R} = \{(t, s)\} \cup \bigcup_{\gamma \in \Gamma} \mathcal{R}^\gamma$$

By construction, the valuation of $N_t$ matches that of $M_s$, and each of the
$\mathcal{R}^\gamma$ are $a$-simulations, hence $\mathcal{R}$ satisfies {\bf
atoms}.

Let $u \in S^N$ and $u' \in S^M$ such that $(u, u') \in \mathcal{R}$. Suppose
that $(u, u') \in \mathcal{R}^\gamma$ for some $\gamma \in \Gamma$. Then as
$\mathcal{R}^\gamma$ is an $a$-simulation, it satisfies {\bf forth-$b$} for
every $b \in A$. Hence for every $v \in uR^N_b$ there exists some $v' \in
u'R^M_b$ such that $(v', v') \in \mathcal{R}^\gamma \subseteq \mathcal{R}$.
Suppose instead that $(u, u') \notin \mathcal{R}^\gamma$ for every $\gamma \in
\Gamma$. Then $(u, u') = (t, s)$. Then for every $v \in tR^N_a$, we note that,
by construction, $v = t^\gamma$ for some $\gamma \in \Gamma$, and so we can take
$s^\gamma \in sR^M_a$, and note that $(t^\gamma, s^\gamma) \in
\mathcal{R}^\gamma \subseteq \mathcal{R}$, by hypothesis (as
$\mathcal{R}^\gamma$ is an $a$-simulation between $N^\gamma_{t^\gamma}$ and
$M_{s^\gamma}$). For every $b \in A - \{a\}$, and every $v \in t^\gamma R^N_b$,
we note that by construction there exists some $\gamma \in \Gamma$ and $v' \in
sR^M_b$ such that $(v', v) \in \mathcal{R}^\gamma \subseteq \mathcal{R}$. Hence
$\mathcal{R}$ satisfies {\bf forth-$b$} for every $b \in A$.

A similar argument to above shows that $\mathcal{R}$ satisfies {\bf forth-$b$}
for every $b \in A - \{a\}$.

Hence $\mathcal{R}$ is an $a$-simulation from $N_t$ to $M_s$. Hence $N_t
\refinement_a M_s$.

Hence $M_s \entails \somerefs_a \covers_a \Gamma$.

Conversely, suppose that $M_s \entails \covers_a \Gamma$. Then there exists a
doxastic model $N_t \refinement_a M_s$, via some $a$-simulation $\mathcal{R}$,
such that $N_t \entails \covers_a \Gamma$. From the definition of the cover
operator, this implies that $N_t \entails \knows_a \bigvee_{\gamma \in \Gamma}
\gamma \land \bigwedge_{\gamma \in \Gamma} \suspects_a \gamma$. In particular we
note that for every $\gamma \in \Gamma$, $N_t \entails \suspects_a \gamma$, and
so there exists some $t^\gamma \in tR^N_a$ such that $N_{t^\gamma} \entails
\gamma$. As $t^\gamma \in tR^N_a$, and $(t, s) \in \mathcal{R}$, by {\bf
forth-$a$} there exists some $s^\gamma \in sR^M_a$ such that $(t^\gamma, s^\gamma)
\in \mathcal{R}$. Hence $\mathcal{R}$ is also an $a$-simulation from
$N_{t^\gamma}$ to $M_{s^\gamma}$, and so $M_{s^\gamma} \entails \somerefs_a
\gamma$, and as $s^\gamma \in sR^M_a$, we also have that $M_s \entails \suspects_a
\somerefs_a \gamma$. As this holds for every $\gamma \in \Gamma$, we finally
have that $M_s \entails \bigwedge_{\gamma \in \Gamma} \suspects_a \somerefs_a
\gamma$.

\paragraph{GMulti} Suppose that $M_s$ is a doxastic model such that $M_s
\entails \covers_b \{ \somerefs_a \gamma \mid \gamma \in \Gamma\}$, where $a \ne
b$ and $\Gamma$ is a finite set of formulae. From the definition of the cover
operator, this implies that $M_s \entails \knows_b \bigvee_{\gamma \in \Gamma}
\somerefs_a \gamma \land \bigwedge_{\gamma \in \Gamma} \suspects_b \somerefs_a
\gamma$. In particular, we note that for every $\gamma \in \Gamma$, there exists
some $s^\gamma \in sR^M_b$ such that $M_{s^\gamma} \entails \somerefs_a \gamma$.
Then there exists a doxastic model $N^\gamma_{t^\gamma} \refinement_a
M_{s^\gamma}$, via some $a$-simulation $\mathcal{R}$, such that
$N^\gamma_{t^\gamma} \entails \gamma$. Without loss of generality, we assume
that the $N^\gamma$ are disjoint.

Let $t$ be a state such that $t \notin S^{N^\gamma}$ for every $\gamma \in
\Gamma$. Then we construct a model $N = (S^N, R^N, V^N)$, where:

\begin{eqnarray*}
S^N &=& \{t\} \cup \bigcup_{\gamma \in \Gamma} S^{N^\gamma}\\
R^N_b &=& \{(t, t^\gamma) \mid \gamma \in \Gamma\} \cup \{(t^\gamma,
t^{\gamma'}) \mid \gamma, \gamma' \in \Gamma\} \cup \bigcup_{\gamma \in \Gamma}
R^{N^\gamma}_b\\
R^N_c &=& \{(t, t') \mid \gamma \in \Gamma, s' \in sR^M_c, t' \in S^N, (s', t')
\in \mathcal{R}^\gamma\} \cup \bigcup_{\gamma \in \Gamma} R^{N^\gamma}_c \text{
for $c \in A - \{b\}$}\\
V^N(p) &=& \begin{cases}
\{t\} \cup \bigcup_{\gamma \in \Gamma} V^{N^\gamma}(p) \text{ if $s \in
V^M(p)$}\\
\bigcup_{\gamma \in \Gamma} V^{N^\gamma}(p) \text{ otherwise}
\end{cases}
\end{eqnarray*}

We note that $N$ is a doxastic model, by similar arguments as used in the proof
for {\bf GKD45}.

We next construct an $a$-simulation $\mathcal{R}$ from $N_t$ to $M_s$, where:

$$\mathcal{R} = \{(t, s)\} \cup \bigcup_{\gamma \in \Gamma} \mathcal{R}^\gamma$$

% TODO

\end{proof}

\section{Decision procedures}
