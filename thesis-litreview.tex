\chapter{Literature review}

We review previous techniques in dynamic epistemic logic and dynamic doxastic
logic, and previous work in the future event logic, that motivates the present
work.

\section{Dynamic epistemic logic}

Dynamic epistemic logic is the study of information change in modal epistemic
systems. Information change is performed by informative updates, that provide
agents with additional information, whilst leaving factual information about the
state of the world unchanged. Notable representations for informative updates
in dynamic epistemic logic include public announcements, epistemic actions, and
action models. I will briefly discuss these representations of informative
updates, and the logics that have been devised to reason about them.

\subsection*{Public announcements}

A public announcement is a simple form of informative update, which can be
thought of as information that is announced to all agents in an epistemic system
at once. A public announcement takes the form of an epistemic formula $\phi$.
The result of a public announcement is that every agent knows $\phi$, every
agent knows that every other agent knows $\phi$, and so on, ad infinitum. In
terms of the Kripke model representation of the epistemic system, a public
announcement has the effect of restricting the states of the Kripke model to
those that are consistent with $\phi$; given the additional information $\phi$,
the states that are not consistent with $\phi$ are now considered inplausible by
every agent, and so these states are removed.

The public announcement logic was introduced and axiomatised by
Plaza~\cite{plaza2007logics}, and more generally by Gerbrandy and
Groenvald~\cite{gerbrandy1997reasoning}. % TODO - More generally?
The logic introduces an operator $[\phi]$, that can be used to reason about the
results of a specific public announcement. The operator, $[\phi] \psi$ means
that after the formula $\phi$ is publicly announced, $\psi$ holds in the
resulting epistemic state. This allows one to reason about the consequences of
specific public announcements in the epistemic system.

Public announcements are a very limited form of informative update, because the
information contained in a public announcement necessarily becomes common
knowledge to all agents in the system. Public announcements cannot for example
model informative updates in which information is provided privately to only
some of the agents in the system. Public announcements are however suited to
some interesting problems, for example the muddy children puzzle that was
discussed in the introduction to this paper.

\subsection*{Epistemic actions}

Epistemic actions capture a more general notion of informative updates than
public announcements. Compared to public announcements, epistemic actions are
able to represent informative updates where information is provided to only a
subset of the agents, or where an agent may be aware that one of several
informative updates may have occurred, but is uncertain as to which specific
informative update has occurred. 

For example, we may have a situation with two agents, $a$ and $b$, where neither
agent knows whether $p$ or $\neg p$, and this fact is common knowledge. The
agent $a$ may be told privately that $p$ is actually the case, and although $b$
cannot overhear this information, $b$ is still aware that some private
communication is taking place. The result of this is that $a$ now knows that
$p$, whilst $b$ continues to not know whether $p$ or $\neg p$. Furthermore,
whereas before the informative update, $b$ knew that $a$ doesn't know whether
$p$ or $\neg p$, after the informative update this is no longer the case; from
the point of view of $b$, the informative update that occurred may have told $a$
that $p$, or it may have told $a$ that $\neg p$, or it may have told $a$
neither of these statements. Whilst the effect of the informative update was to
provide additional knowledge to $a$, it also provided additional uncertainty to
$b$.

Epistemic actions are represented as formulae, built from epistemic formulae
using special operators. The effect of the operators is to restrict the effect
of epistemic actions to specific subsets of agents, or to signify that a subset
of agents is aware that one of several epistemic actions may have occurred, but
the agents are uncertain as to which one actually occurred. Epistemic actions
are capable of representing any public announcement.

The logic of epistemic actions is an extension of epistemic logic, which was
introduced by van
Ditmarsch~\cite{vanditmarsch1999logic,vanditmarsch2001knowledge,vanditmarsch2007dynamic}.
As in the public announcement logic, a new operator is introduced that has the
effect of performing an epistemic action. The operator $[\phi] \psi$ means that
after the epistemic action $\phi$ is performed, $\psi$ holds in the resulting
epistemic state. As epistemic actions can represent any public announcement, the
logic of epistemic actions is a generalisation of the public announcement logic.

\subsection*{Action models}

Action models are a representation of informative update that are similar in
power to epistemic actions. Unlike public announcements and epistemic actions,
which are both represented by formulae, an action model is represented by a
graph. Nodes on the action model are labelled with epistemic formulae, known as
preconditions, and edges are labelled by agents. An action model is {\em
executed} on a Kripke model by performing an action similar to a Cartesian
product, and restricting the result according to the preconditions and agents
that label the action model. 

Similar to an epistemic action, executing an action model on a Kripke model may
result in different information being provided to different agents in the
system, and in some cases additional uncertainty being introduced. Action models
are capable of representing any public announcement.

The logic of action models was introduced by Baltag, Solecki and
Moss\cite{baltag2004logics}. Again, an operator is introduced that has the
effect of performing an informative update, this time in the form of executing an
action model on the Kripke model that represents the current epistemic state.
The operator, $[M,s]\psi$ means that after the action model $(M, s)$ is
executed, $\psi$ holds in the resulting epistemic state. As epistemic actions
are capable of representing any public announcement, the action model logic is
a generalisation of public announcement logic.

Action models and epistemic actions can mostly represent the same kinds of
informative updates, however are not equivalent. In general it is possible to
translate between action models and epistemic actions, but there are exceptions
to this. % TODO - clarify
Both kinds of informative updates are interesting for their own reasons;
epistemic actions provide a representation of informative updates that give a
more intuitive description of the informative update, but applying an
epistemic action to a Kripke model is not intuitive; action models on the other
hand are relatively simple to execute on Kripke models, however it is not
intuitive how a particular action model represents a particular informative
update. There some informative updates that may only be represented using one of
the representations and not the other.

\subsection*{Arbitrary public announcement logic and arbitrary event model
logic}

Balbiani, Baltag, van Ditmarsch, et al.\cite{balbiani2007arbitrary} introduced
the arbitrary public announcement logic, which is an extension of the public
announcement logic, providing quantification over arbitrary public
announcements. It introduces an operator, $\Box\psi$, which means that after any
public announcement, $\psi$ holds. Its dual operator, $\Diamond\psi$ means that
after some public announcement, $\psi$ holds.

The same paper briefly discusses a possible arbitrary event model logic, which
is a generalisation of arbitrary public announcement logic, allowing for any
kind of informative update, modelled as the execution of an action model. French
and van Ditmarsch~\cite{french2008undecidability} later showed that the
arbitrary public announcement was undecidable. Whilst the arbitrary event model
logic has not been considered in depth, the undecidability result in arbitrary
public announcement logic has encouraged research into weaker versions of this
logic instead, as they are more likely to be decidable.

\section{Dynamic doxastic logic}

Dynamic doxastic logic is the study of information change in modal doxastic
systems. In dynamic epistemic logic, we consider how knowledge changes in
response to informative updates which provide additional information. A result
of this is that after an agent has learned some positive knowledge, the agent
continues to have that positive knowledge; further informative updates cannot
cause that agent to {\em forget} positive knowledge that it already has. By
contrast, in a doxastic system, it is reasonable for an agent to lose a belief
that it once had, if it learns that the belief is no longer founded. This is
represented in {\em belief revision}.

\subsection{Belief revision}

% TODO - everything

\section{Other related logics}

\subsection{Description logics}

\subsection{Propositional dynamic logics}

\subsection{Bisimulation quantified logics}

\section{Future event logic}

van Ditmarsch and French~\cite{french2009simulation} introduced the future event
logic, an extension of epistemic logic that provides an operator for
quantification over arbitrary refinements of Kripke models. van Ditmarsch and
French provide the semantics of the logic, and provide several results and
comparisons to justify the future event logic as accurately capturing the notion
of quantifying over arbitrary informative updates. van Ditmarsch and French also 
compare the semantics of the future event logic to previously defined logics, in
particular to the bisimulation quantified epistemic logic, and to the proposed
arbitrary event model logic.

The future event logic is related to the bisimulation quantified epistemic
logic, described by French~\cite{french2006bisimulation}, as refinements and
bisimulations are related concepts. The refinement quantifier introduced in the
future event logic can be viewed as a weaker version of the bisimulation
quantifiers in bisimulation quantified epistemic logic. However the two logics
have completely different interpretations and applications with respect to
epistemic logic. A notable difference between the two is that bisimulation
quantifiers quantify over the Kripke models that are bisimilar except for a
given propositional atom $p$, thus the quantifier binds the proposition $p$ as a
variable. This is not the case with the refinement quantifier of the future
event logic. % TODO - applications?

van Ditmarsch and French show that the finite refinements of a Kripke
model are equivalent to the results of executing some action model, and
vice-versa, thus quantifying over refinements is equivalent to quantifying over
the results of action model executions. The future event logic is also compared
to a possible arbitrary event model logic. The main difference between the two
logics is that the arbitrary event model logic also has an operator for
reasoning about the result of a specific action model execution. In fact, van
Ditmarsch and French~\cite{french2009simulation} show that if such an operator
is added to the future event logic, then the resulting logic is equivalent to
the arbitrary event model logic. The semantics of the refinement quantifier in
the future event logic are much simpler than the semantics of the action model
quantifier in arbitrary event model logic, as they do not rely on the mechanics
of action models.

The future event logic is considered in more depth by French, Pinchinat and van
Ditmarsch~\cite{french2010future}. They consider the future event logic as an
extension of single-agent modal logic, rather than considering it as an
extension of epistemic logic, or considering the multi-agent logic.  An
axiomatisation for the logic, and a tableau method for solving the
satisfiability problem are provided. The axiomatisation is shown to be sound and
complete, with the completeness proof performed by a syntactic translation of
formulae in the future event logic into semantically provably equivalent
formulae in modal logic. This translation shows that the future event logic is
no more expressive than modal logic, although a result is given that shows that
the future event logic is exponentially more succinct than modal logic (that is,
some formulae in the future event logic can only be expressed in modal logic by
a formula that is exponentially longer). The expressivity result allows one to
derive decidability results from those results in modal logic, although a
decision procedure is given directly in the setting of the future event logic,
using a tableau method, and this provides an upper bound for the complexity of
the logic. Extending the future event logic to multi-agent modal logic, and to
epistemic and doxastic logics are left as future work.
