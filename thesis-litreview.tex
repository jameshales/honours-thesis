\chapter{Literature review}

\section{Introduction}

Informative updates can be loosely defined as actions performed in an epistemic
system that result in new information being provided to agents in the system.
The field of dynamic epistemic logic considers how the knowledge of a collection
of agents may change as a result of such informative updates. Informative
updates may provide agents with increased certainty about facts about the world,
but they may also increase uncertainty about the knowledge of other agents. The
former typically happens when the informative update takes the form of an
announcement or message to a particular agent, which makes a statement
containing facts about the world. The latter typically happens when, from the
point of view of the agent, it is clear that some informative update occurred,
such as a private message to another agent, but it is not known exactly which
informative update occurred.

Previous work in dynamic epistemic logic has considered various logical systems
in which informative updates can be modelled and reasoned about. Particularly of
interest to my research area, the future event logic, are logics that introduce
operators representing the application of specific informative updates, or
operators representing quantification over informative updates.

\section{Review of previous techniques in dynamic epistemic logic}

Techniques in dynamic epistemic logic have been based on a variety of
definitions of informative updates. Important definitions for modelling
informative updates include public announcements, epistemic actions, action
models, and refinements of Kripke models. I will briefly introduce these
different definitions of informative updates, and the logics that have been
based on them. Typically each definition of informative update comes with a
method for applying the update to an epistemic system, for reasoning about the
consequences of the update, and for quantifying over arbitrary informative
updates.

\subsection*{Public announcements}

A public announcement is a simple form of informative update, in which
information is announced to all agents in an epistemic system at once.  The
information that is announced takes the form of an epistemic formula. The result
of a public announcement is that the information that is announced becomes
common knowledge, which means that as a result of the announcement, all of the
agents in the system know the information, all of the agents know that all
agents know the information, all of the agents know that all agents know that
all agents know the information, and so on. In terms of the Kripke model
representation of the epistemic system, this has the effect of restricting the
Kripke model to only those possible worlds which are consistent with the
announced formula, removing those possible worlds which are inconsistent with
the announcement. As such it is a very simple matter to apply a public
announcement to a Kripke model.

The public announcement logic was introduced and axiomatised by
Plaza\cite{plaza2007logics} and more generally by Gerbrandy and
Groenvald\cite{gerbrandy1997reasoning}. The logic introduces an operator that
has the effect of performing a public announcement to all agents in the
epistemic system. The operator, $[\phi]\psi$ takes an epistemic formula $\phi$,
and a public announcement logic formula $\psi$, and states that after $\phi$ is
publicly announced, $\psi$ holds. This allows one to reason about the
consequences of the public announcement in the epistemic system.

Public announcements are a very limited form of informative update, because the
result of a public announcement is necessarily common knowledge to all of the
agents in the system. Public announcements cannot model informative updates in
which the result of the update is only known to some agents in the system, or in
which the update introduces new uncertainty to some agents. Examples of such
informative updates are private messages, that only the recipient of the message
knows the contents of the message, and other agents may be aware of the message,
but are uncertain as to its contents.

\subsection*{Epistemic actions}

Epistemic actions capture a more sophisticated notion of an informative update
than public announcements. Epistemic actions are formulae that are built using
operators that restrict the result of a particular epistemic action to a subset
of the agents, or that signify that one of several possible epistemic actions
may have occurred, but from the point of view of some agents it is uncertain
which action actually occurred. This allows epistemic actions to model
informative updates in which not all agents know of the results of the
informative update, or in which the informative update results in further
uncertainty, something that is not possible with public announcements.

The logic of epistemic actions is a generalisation of the public announcement
logic, which was pioneered by van
Ditmarsch\cite{vanditmarsch1999logic}\cite{vanditmarsch2001knowledge}\cite{vanditmarsch2007dynamic}.
As in the public announcement logic, a new operator is introduced that has the
effect of performing an epistemic action.  The operator, $[\phi]\psi$ takes an
epistemic action $\phi$ and an epistemic action logic formula, $\psi$, and
states that after the epistemic action $\phi$ is performed, $\psi$ holds.

\subsection*{Action models}

Action models provide a representation of informative updates that is similar
in its capabilities to epistemic actions. Unlike public announcements and
epistemic actions, which are both represented by formulae, an action model is
represented by a graph. Nodes in an action model are labelled by epistemic
formulae, known as preconditions. The edges in an action model are labelled by
agents from the epistemic system, much in the way that edges are labelled in an
epistemic Kripke model.

An action model is applied to a Kripke model by restricting the Cartesian
product of the two graphs according to the preconditions and adjacencies in the
action model.  Applying an action model to a Kripke model may result in possible
worlds from the Kripke model being duplicated, having the effect that the
results of the informative update may be uncertain to some agents. The
restriction applied to the Cartesian product may result in different agents
receiving different information from the informative update.

The logic of action models was introduced by Baltag, Solecki and
Moss\cite{baltag2004logics}. Again, an operator is introduced that has the
effect of performing an informative update, this time in the form of applying an
action model to the Kripke model that represents the current epistemic system.
The operator, $[M,s]\psi$ takes a pointed action model, $(M, s)$, and an action
model logic formula $\psi$, and states that after the action model $(M, s)$ is
applied, $\psi$ holds.

Action models and epistemic actions can mostly represent the same kinds of
informative updates. In general it is possible to translate between action
models and epistemic actions, however there are some exceptions to this. Both
kinds of informative updates are interesting for their own reasons; epistemic
actions describe informative updates intuitively, but applying them is not as
easy, whereas action models provide an algorithmic approach to applying an
informative update, however it is not as intuitive how a particular action model
represents a particular informative update. There are of course kinds of
informative updates that may only be represented using one of the
representations and not the other.

\subsection*{Arbitrary public announcement logic and arbitrary event model
logic}

Balbiani, Baltag, van Ditmarsch, et al.\cite{balbiani2007arbitrary} introduced
the arbitrary public announcement logic, which is an extension of the public
announcement logic, providing quantification over arbitrary public
announcements. It introduces an operator, $\Box\psi$, which means that after any
public announcement, $\psi$ holds. Its dual operator, $\Diamond\psi$ means that
after some public announcement, $\psi$ holds.

The same paper briefly discusses a possible arbitrary event model logic, which
is a generalisation of arbitrary public announcement logic, allowing for any
kind of informative update, via the application of an action model update. This
is interesting because it attempts to do what the future event logic does,
however with some differences (notably, it also has an operator for applying
specific informative updates, which the future event logic does not have).

French and van Ditmarsch\cite{french2008undecidability} later showed that the
arbitrary public announcement logic was undecidable for the satisfiability
problem, and therefore by extension, the arbitrary event model logic is also
undecidable.

\section{Review of future event logic}

French and van Ditmarsch\cite{french2009simulation} introduced the future event
logic, which is an extension of epistemic logic that provides an operator for
quantification over refinements of Kripke models. French and van Ditmarsch
provide the semantics for the logic, and show that quantifying over refinements
of Kripke models is equivalent to quantifying over arbitrary action model
updates, up to the limits of action models (as action models are necessarily
finite, whereas refinements are not).

The paper compares the semantics of the future event logic to previously defined
logics and structures built on epistemic logic, in order to show that it
faithfully captures the idea of informative updates to epistemic systems. In
particular, comparisons are made to the bisimulation quantified epistemic logic,
and to the arbitrary event model logic. 

The future event logic is related to the bisimulation quantified epistemic
logic, described by French\cite{french2006bisimulation}, as bisimulations and
refinements are related ideas. The refinement quantifier introduced in the
future event logic can be viewed as a weaker version of the bisimulation
quantifiers. However, the two logics have completely different interpretations
and applications with respect to epistemic logic.

The arbitrary event model logic is conceptually similar to the future event
logic, in the sense that it tries to represent the same notions. The main
difference between the two logics is that the arbitrary event model logic has an
operator for applying a specific action model update, whereas the future event
logic does not. French and van Ditmarsch\cite{french2009simulation} in fact show
that if such an operator is added to the future event logic, then the resulting
logic is equivalent to the arbitrary event model logic.

The future event logic is discussed further in depth by French, Pinchinat and
van Ditmarsch\cite{french2010future}, where an axiomatisation of the logic, and
a tableau method for solving the satisfiability problem is presented. The
axiomatisation is shown to be sound and complete, and interestingly the
completeness proof is done via a syntactic reduction of arbitrary formulae in
the future event logic to formulae in the epistemic logic. This reduction shows
that the future event logic is no more expressive than the epistemic logic
(although it is more succinct), and also proves that the future event logic is
decidable in the satisfiability problem, a useful property that is not shared
with the arbitrary event model logic. A proof is given to show that the future
event logic is exponentially more succinct than the epistemic logic, a fact
which is used to show that the tableau method presented in the paper is likely
to have optimal algorithmic complexity.

The version of the future event logic that is axiomatised by French, Pinchinat
and van Ditmarsch\cite{french2010future} was an extension of the single-agent
epistemic logic K, which is a normal modal logic over the set of all Kripke
models. Conventional epistemic logic works in S5, the set of reflexive,
transitive and symmetric Kripke models. French, Pinchinat and van Ditmarsch
briefly discuss the possibility of extending their axiomatisation to S5 and
KD45, although leave this for future research.

My area of research in the future event logic will focus on a review of the
results presented in French, Pinchinat and van Ditmarsch's latest
paper\cite{french2010future}, including of their axiomatisation, completeness
proof, and solution to the satisfiability problem. It will include an
implementation and analysis of their results. My research will also explore
axiomatising the logic in S5, which will likely take from the axiomatisation,
soundness and completeness proofs presented by French, et al. 
