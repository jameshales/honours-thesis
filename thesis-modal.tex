\chapter{Modal, doxastic and epistemic logics}\label{modal}

In this appendix we give a brief introduction to modal, doxastic and epistemic
logics, the required background material for this paper.

To introduce modal logic, we will begin with the definition of a Kripke model.

% Definition of Kripke models
Let $A$ be a non-empty, finite set of agents, and let $P$ be a non-empty,
countable set of propositional atoms.

\begin{definition}[Kripke model]
A \textit{Kripke model} $M = (S, R, V)$ consists of a \textit{domain} $S$, which
is a set of states (or worlds), an \textit{accessibility relation} $R : A \to
\mathcal{P}(S \times S)$, and a \textit{valuation} $V : P \to \mathcal{P}(S)$. 

The class of all Kripke models is called \classK{}. We write $M \in \classK{}$
to denote that $M$ is a Kripke model.
\end{definition}

For $R(a)$, we write $R_a$. We write $sR_a$ for $\{t \mid (s, t) \in R_a\}$ and
we write $R_at$ for $\{s \mid (s, t) \in R_a\}$. As we will be required to
discuss several models at once, we will use the convention that $M = (S^M, R^M,
V^M)$, $N = (S^N, R^N, V^N)$, and so on. For $s \in S^M$ we will let $M_s$ refer
to the pair $(M, s)$, also known as the pointed Kripke model of $M$ at state
$s$.

Modal logics can be said to model notions such as necessity, knowledge or
belief. To help with the introduction however, we will for now adopt an analogy
using knowledge. Under the {\em possible worlds} interpretation of modal logic,
each agent can be said to live in a world where there are a certain set of facts
that are true or false. The agent may not have complete knowledge about which of
those facts are true or false, and so it may consider several possibilities. We
say that an agent considers several possible worlds, where the truth of each
fact may be different in each world. Thus each agent considers a set of worlds
to be possible, and we say that the agent {\em knows} a statement about the
world to be true if that statement is true in all of the worlds that it
considers possible. For example, if an agent knew everything that there was to
know about the world, then it would consider only one world possible: the actual
world; whereas if an agent was uncertain about a particular statement $\phi$
about the world, then the agent would consider at least two worlds possible: a
world where $\phi$ is true and a world where $\phi$ is false.

Kripke models are an encoding of this possible worlds interpretation. Each state
of the Kripke model represents a world, and the facts that are true at that
world are represented by the propositional atoms that are true at the state,
according to the valuation of the Kripke model. At each state, the worlds that
an agent considers possible are those worlds that are accessible from that state
according to that agent's accessibility relation.

It should be emphasised that the worlds that an agent considers possible are not
a global property of the Kripke model, but rather the possible worlds can vary
from state to state. This is reflected in the fact that the possible worlds are
represented by the states accessible from a particular state, rather than by a
fixed subset of states of the Kripke model. Furthermore it should be noted that
the statements about the world that an agent can know about is not limited just
to statements directly about the facts of the world; agents may be aware of the
knowledge of other agents.

In the basic modal logic, we usually refer to the notion of necessity. Thus if a
statement is true in all of the worlds that an agent considers possible, then
the agent considers that statement to be necessary. We will explain the
difference between necessity, knowledge and belief later; for the moment the
difference is not important.

We will now give a precise definition of the language, \lang{}, of basic modal
logic, and its semantics over a general class of Kripke models.

\begin{definition}[Language of \lang{}]
Given a non-empty, finite set of agents $A$ and a non-empty, countable set of
propositional atoms $P$, the language of \lang{} is defined by the following
abstract syntax:
$$
\alpha ::=  p \bnfalt
            \neg \alpha \bnfalt
            \alpha \land \alpha \bnfalt
            \knows_a \alpha
$$
where $p \in P$, $a \in A$ and $\alpha \in \lang{}$.
\end{definition}

Standard abbreviations include:
$\top ::= p \lor \neg p$ for some $p \in P$;
$\bot ::= \neg \top$;
$\phi \lor \psi ::= \neg (\neg \phi \land \neg \psi)$;
$\phi \implies \psi ::= \neg \phi \lor \psi$;
and $\suspects_a \phi ::= \neg \knows_a \neg \phi$.

The symbol $\knows_a$ is said to represent necessity in modal logics. Thus we
may read $\knows_a \phi$ as ``$a$ considers $\phi$ necessary''. The dual
operator, $\suspects_a$ represents possibility; hopefully it is intuitive that
if a statement is not necessarily false, then it is possible. Thus we may read
$\suspects_a \phi$ as ``$a$ considers $\phi$ possible''.

\begin{definition}[Semantics of \logicC{}]
Let \classC{} be a class of Kripke models, and let $M = (S, R, V) \in \classC$
be a Kripke model taken from \classC{}. The interpretation of $\phi$ in a
pointed Kripke model $M_s$ is defined inductively:
\begin{eqnarray*}
M_s &\entails& p \text{ iff } s \in V_p\\
M_s &\entails& \neg \phi \text{ iff } M_s \nentails \phi\\
M_s &\entails& \phi \land \psi \text{ iff } M_s \entails \phi \text{ and } M_s
\entails \psi\\
M_s &\entails& \knows_a \phi \text{ if for all } t \in S : (s, t) \in R_a \text{
implies } M_t \entails \phi
\end{eqnarray*}
\end{definition}

We say that a formula $\phi$ is {\em satisfied} by a pointed Kripke model $M_s
\in \classC$ if and only if $M_s \entails \phi$. We say that $\phi$ is satisfied
by a Kripke model $M \in \classC$ if and only if $M_s \entails \phi$ for some $s
\in S^M$. We say that $\phi$ is {\em satisfied} by a class of Kripke models
$\classC$ if and only if it is satisfied by every Kripke model $M \in \classC$.
We say that $\phi$ is {\em valid} in a Kripke model $M \in \classC$ if and only
if $M_s \entails \phi$ for every $s \in S^M$. We write $M \entails \phi$. We say
that $\phi$ is {\em valid} in a class of Kripke models $\classC$ if and only if
$M \entails \phi$ for every $M \in \classC$. We write $\entails_{\classC} \phi$.

We take a moment to remark on the interpretation of the modal logic. We note
that modal formulae are interpreted in pointed Kripke models. In our possible
worlds interpretation, a pointed Kripke model $M_s$ can be said to be a Kripke
model where we have designated a particular state, $s$, as the actual or current
world. A propositional variable is interpreted with respect to the valuation at
the current state of the Kripke model. Negation and conjunction are interpreted
in the intuitive manner, both at the current state. The most notable aspect of
the interpretation of modal logic is the interpretation of the $\knows_a$
operator.  The $\knows_a$ operator effectively moves consideration from the
current state of the pointed Kripke model to each of the states accessible from
the current state, under the Kripke model's accessibility relation. This
captures the notion that we described earlier, that an agent considers a
statement to be necessary if that statement is true in all of the worlds that it
considers possible. It should be clear that under this interpretation, the dual
operator, $\suspects_a$, also captures the notion that an agent considers a
statement to be possible if that statement is true in at least one of the worlds
that it considers possible.

Modal logics are interpreted over classes of Kripke models. The simplest normal
modal logic, \logicK{}, is interpreted over the class of all Kripke models,
\classK{}. Variants of modal logic are interpreted over subclasses of \classK{},
which impose constraints on the accessibility relations of the Kripke models.
Doxastic logic, \logicKD{}, and epistemic logic, \logicS{}, are two variants of
modal logic that are interpreted over different classes of epistemic models: the
class of doxastic models, \classKD{}, and the class of epistemic models,
\classS{}, respectively. As we will see later, the constraints that are imposed
on these classes of models gives us properties that make doxastic and epistemic
logic represent simple notions that we have about belief and knowledge.

We will now define doxastic and epistemic models, first defining the relational
properties that we will use to describe them.

\begin{definition}[Relational properties]
Let $S$ be a set and let $R \subseteq S \times S$. Then we define the following
properties of $R$.
\begin{enumerate}
\item $R$ is {\em serial} if and only if for every $s \in S$ there
exists some $t \in S$ such that $(s, t) \in R$.
\item $R$ is {\em reflexive} if and only if for every $s \in S$ we
have that $(s, s) \in R$.
\item $R$ is {\em transitive} if and only if for every $(s, t), (t, r)
\in R$ we also have that $(s, r) \in R$.
\item $R$ is {\em Euclidean} if and only if for every $(s, t), (s, r)
\in R$ we also have that $(t, r) \in R$.
\item $R$ is {\em symmetric} if and only if for every $(s, t) \in R$
we also have that $(t, s) \in R$.
\end{enumerate}
\end{definition}

We note that if $R$ is reflexive then it is also serial. We also note that 
$R$ is reflexive and Euclidean, if and only if it is reflexive, transitive and
symmetric.

\begin{definition}[Doxastic model]
A \textit{doxastic model} is a Kripke model $M = (S, R, V)$ such that the
relation $R_a$ is serial, transitive, and Euclidean for all $a \in A$. The class
of all doxastic models is called \classKD{}. We write $M \in \classKD{}$ to
denote that $M$ is a doxastic model.
\end{definition}

\begin{definition}[Epistemic model]
An \textit{epistemic model} is a Kripke model $M = (S, R, V)$ such that the
relation $R_a$ is reflexive, transitive and symmetric for all $a \in A$. The
class of all epistemic models is called \classS{}. We write $M \in \classS{}$ to
denote that $M$ is an epistemic model.
\end{definition}

We note that every epistemic model is also a doxastic model, as if a relation
$R_a$ is reflexive, transitive and symmetric, then it is also serial and
Euclidean, and therefore satisfies the constraints of a doxastic model.

The logics \logicK{}, \logicKD{} and \logicS{} are instances of \logicC{} with
classes \classK{}, \classKD{} and \classS{} respectively. It should be noted
that \logicKD{} is a conservative extension of \logicK{}, and \logicS{} is a
conservative extension of \logicKD{} (and also of \logicK{}). This means that
every valid formula in \logicK{} is also valid in \logicKD{}, and likewise for
\logicKD{} and \logicS{}. This is because any formula that is valid with respect
to a particular class of Kripke models is also valid for any subclass of those
Kripke models.

\begin{proposition}[Properties of \logicK{}, \logicKD{} and
\logicS{}]\label{pre-properties}
Let $\phi, \psi \in \lang$. Then the following are valid:

\begin{enumerate}
\item\label{pre-property-k} $\entails_{\classK} \knows_a (\phi \implies \psi) \implies \knows_a \phi
\implies \knows_a \psi$
\item\label{pre-property-conj} $\entails_{\classK} \knows_a \phi \land \knows_a \psi \iff \knows_a (\phi \land \psi)$
\item $\entails_{\classK} \suspects_a \phi \lor \suspects_a \psi \iff \suspects_a (\phi
\lor \psi)$
\item\label{pre-property-d} $\entails_{\classKD} \knows_a \phi \implies \suspects_a \phi$
\item\label{pre-property-4} $\entails_{\classKD} \knows_a \phi \implies \knows_a \knows_a \phi$
\item\label{pre-property-5} $\entails_{\classKD} \neg\knows_a \phi \implies \knows_a \neg \knows_a \phi$
\item\label{pre-property-t} $\entails_{\classS} \knows_a \phi \implies \phi$
\end{enumerate}
\end{proposition}

\begin{proof}
\begin{enumerate}
\item Let $M_s \in \classK{}$ such that $M_s \entails \knows_a (\phi \implies \psi)$
and $M_s \entails \knows_a \phi$. Then for every $t \in sR_a$, we have $M_t
\entails \phi \implies \psi$ and $M_t \entails \phi$. Therefore for every $t \in sR_a$,
we have $M_t \entails \psi$, and therefore we have that $M_s \entails \knows_a \psi$.
\item Exercise.
\item Exercise.
\item Let $M_s \in \classKD{}$ such that $M_s \entails \knows_a \phi$. Then for
every $t \in sR_a$, we have $M_t \entails \phi$. As $M_s \in \classKD{}$, we have
that $R_a$ is serial. Therefore $sR_a \neq \emptyset$, and so there exists some
$t \in sR_a$ such that $M_t \entails \phi$. Therefore $\suspects_a \phi$.
\item Let $M_s \in \classKD{}$ such that $M_s \entails \knows_a \phi$. Then
consider $t \in sR_a$ and $t' \in tR_a$. As $M_s \in \classKD{}$, we have that
$R_a$ is transitive. Therefore $t' \in sR_a$, and so $M_{t'} \entails \phi$.
Therefore $M_t \entails \knows_a \phi$, and so $M_s \entails \knows_a \knows_a \phi$. 
\item Let $M_s \in \classKD{}$ such that $M_s \entails \neg \knows_a \phi$. Then
there exists some $t \in sR_a$ such that $M_t \entails \neg \phi$. Then consider
$t' \in sR_a$. As $M_s \in \classKD{}$, we have that $R_a$ is Euclidean.
Therefore $t \in t'R_a$, and so $M_t \entails \neg \knows_a \phi$. Therefore $M_s
\entails \knows_a \neg \knows_a \phi$.
\item Let $M_s \in \classS{}$ such that $M_s \entails \knows_a \phi$. Then for
every $t \in sR_a$, we have $M_t \entails \phi$. As $M_s \in \classS{}$, we have
that $R_a$ is reflexive. Therefore $s \in sR_a$, and so $M_s \entails \phi$.
\end{enumerate}
\end{proof}

We note that the properties described in Proposition~\ref{pre-properties}
capture intuitive properties of belief and knowledge that lead to doxastic and
epistemic logics being referred to as the logics of belief and knowledge. The
property (\ref{pre-property-k}) is commonly known as the {\em distribution
axiom}, and ensures that an agent knows all logical consequences of its own
knowledge.  This property is also sometimes referred to as logical omniscience.
The property (\ref{pre-property-d}) is commonly known as the {\em consistency
axiom}, and ensures that an agent cannot have inconsistent beliefs, i.e. beliefs
that contradict one another.  The related, but stronger property,
(\ref{pre-property-t}) is commonly known as the {\em truth axiom}, and ensures
that an agent can only know statements that are actually true in the current
world. The properties (\ref{pre-property-4}), and (\ref{pre-property-5}) are
commonly known as the {\em positive introspection axiom} and the {\em negative
introspection axiom} respectively, and ensures that if an agent knows (or
believes) a statement, then they know that they know it (or believe that they
believe it), and likewise if they do not know or believe a statement, then they
know or believe that this is the case.

We note that the proofs of properties (\ref{pre-property-d}),
(\ref{pre-property-4}) and (\ref{pre-property-5}) rely only on the serial,
transitive and Euclidean properties of \classKD{} models, respectively. We also
note that the proof of property (\ref{pre-property-t}) relies only on the
reflexive property of \classS{} models. In fact, the class of Kripke models that
satisfy each of these properties are precisely the class of Kripke models that
satisfy the corresponding relational properties we have described; for example,
the class of Kripke models that satisfy property (\ref{pre-property-d}) is
precisely the class of models that have the serial property, and the class of
Kripke models that satisfy property (\ref{pre-property-4}) is precisely the
class of models that have the transitive property. This demonstrates a
connection between the structural constraints imposed upon doxastic and
epistemic models, and the properties we have described above, and we will refer
back to this connection shortly.

We will now provide a Hilbert-style axiomatisation for \logicK{}, \logicKD{} and
\logicS{}. A Hilbert-style axiomatisation gives a method for deriving valid
formulae in a logic. The type of axiomatisation that we give is called a
substitution schema, which comprises of a set of axioms and a set of rules. The
axioms are statements that contain variables, and substituting the variables for
well-formed formulae gives a valid statement in the logic. The rules provide a
method for deriving a validity from some other validities. The main results of
this paper are axiomatisations of this style for the variants of refinement
quantified modal logics.
 
\begin{definition}[Axiomatisation \axiomK{}]\label{pre-axiom-k}
The axiomatisation \axiomK{} is a substitution schema consisting of the
following axioms:
$$
\begin{array}{rl}
{\bf P} & \text{All propositional tautologies}\\
{\bf K} & \knows_a (\phi \implies \psi) \implies \knows_a \phi \implies \knows_a \psi\\
\end{array}
$$

Along with the rules:
$$
\begin{array}{rl}
{\bf MP} & \text{From $\proves \phi \implies \psi$ and $\proves \phi$, infer
$\proves \psi$}\\
{\bf NecK} & \text{From $\proves \phi$ infer $\proves \knows_a \phi$}\\
\end{array}
$$
\end{definition}

This axiomatisation is shown to be sound and complete by Hughes and
Creswell~\cite{hughes1996new}.

We say that a formula $\phi$ is {\em provable} or {\em derivable} under an
axiomatisation if and only if it can be derived using some finite sequence of
axioms and rules from that axiomatisation, and we write $\proves \phi$. When we
are discussing multiple axiomatisations at once, we may add a subscript to the
turnstile symbol, e.g. $\proves_\axiomK \phi$, to make it more explicit which
logic we are referring to.

An axiomatisation of a logic must have two important properties: {\em soundness}
and {\em completeness}. An axiomatisation is sound with respect to a logic if
and only if every formula that can be derived from the axiomatisation is also
valid in that logic. An axiomatisation is complete if and only if every formula
that is valid in the logic is also derivable.

The axiomatisations for \logicKD{} and \logicS{} are extensions of the
axiomatisation \axiomK{}, which we will now provide.

\begin{definition}[Axiomatisation \axiomKD{}]
The axiomatisation \axiomKD{} is a substitution schema consisting of the axioms
and rules of \axiomK{}, along with the additional axioms:
$$
\begin{array}{rl}
{\bf D} & \knows_a \phi \implies \suspects_a \phi\\
{\bf 4} & \knows_a \phi \implies \knows_a \knows_a \phi\\
{\bf 5} & \suspects_a \phi \implies \knows_a \suspects_a \phi\\
\end{array}
$$
\end{definition}

We note that the additional axioms for \axiomKD{} correspond to the properties
of \classKD{} models described in Proposition~\ref{pre-properties}. The axiom of
{\bf 5} is equivalent to property (\ref{pre-property-5}) from
Proposition~\ref{pre-properties}. The axioms {\bf D}, {\bf 4} and {\bf 5} serve
to restrict the logic to considering only the Kripke models that are serial,
transitive and Euclidean.

This axiomatisation is shown to be sound and complete by
Gabbay~\cite{gabbay2003many}.

\begin{definition}[Axiomatisation \axiomS{}]
The axiomatisation \axiomS{} is a substitution schema consisting of the axioms
and rules of \axiomK{}, along with the additional axioms:
$$
\begin{array}{rl}
{\bf T} & \knows_a \phi \implies \phi\\
{\bf 5} & \suspects_a \phi \implies \knows_a \suspects_a \phi\\
\end{array}
$$
\end{definition}

This axiomatisation is shown to be sound and complete by Hughes and
Creswell~\cite{hughes1996new}.

Once again, we note that the additional axioms for \axiomS{} correspond to the
properties of \classS{} models described in Proposition~\ref{pre-properties}.
The axioms {\bf T} and {\bf 5} serve to restrict the logic to considering only
the Kripke models that are reflexive and Euclidean.  We note that the reflexive
and Euclidean properties together imply transitivity and symmetry. Although it
is not immediately obvious, the axioms {\bf D} and {\bf 4} from doxastic logic
are both derivable using the axiomatisation \axiomS{}.

Finally we remark on the differences between the notions of belief and knowledge
that are modelled by doxastic and epistemic logics. As we have seen, doxastic
and epistemic logics both have a notion of positive introspection and negative
introspection. This is reflected by the transitive and Euclidean properties of
\classKD{} and \classS{} models. These properties are shown to be defining of
these logics by their representations in the axiomatisations \axiomKD{} and
\axiomS{}, as the axioms {\bf 4} and {\bf 5}. The differences between the logics
are due to the differences between the consistency axiom and the truth axiom.
This is reflected by the serial property of \classKD{} models, and by the
reflexive property of \classS{} models, and these properties are shown to be
defining of these logics by the axioms {\bf D} and {\bf T} respectively. Thus we
can simply say that knowledge is true belief, and that belief is like knowledge,
with the exception that beliefs do not necessarily have to be true, but they do
have to be consistent.
