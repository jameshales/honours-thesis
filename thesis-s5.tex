\chapter{Epistemic logic}

\section{Syntax and semantics}

Here we define the syntax and semantics of the logic \logicSF{}, which
restricts the logic \logicKF{}, as defined by van Ditmarsch and
French~\cite{french2009simulation}, to deal with only models and refinements of
epistemic models.

% TODO - more motiviation

\begin{definition}[Language of \langF{}] % TODO - move to the K section
Given a finite set of agents $A$ and a set of propositional atoms $P$, the
language of \langF{} is defined by the following abstract syntax:

$$
\phi ::=    p \bnfalt
            \neg \phi \bnfalt
            \phi \land \phi \bnfalt
            \knows_a \phi \bnfalt
            \allrefs_a \phi
$$
where $a \in A$ and $p \in P$.
\end{definition}

Standard abbreviations include:
$\top ::= \phi \lor \neg \phi$;
$\bot ::= \neg \top$;
$\phi \lor \psi ::= \neg (\neg \phi \land \neg \psi)$;
$\phi \implies \psi ::= \neg \phi \lor \psi$;
and $\suspects_a \phi ::= \neg \knows_a \neg \phi$.
We use an abbreviation for the dual of the $\allrefs_a$ operator,
$\somerefs_a \phi ::= \neg \allrefs_a \neg \phi$.

We also use the cover operator $\covers_a \Gamma$, where $\Gamma$ is a finite
set of formulae, which is an abbreviation for 
$\covers_a \Gamma ::= \knows_a \bigvee_{\gamma \in \Gamma} \gamma \land
\bigwedge_{\gamma \in \Gamma} \suspects_a \gamma$. The cover operator is relied
on for our axiomatisation, in much the same way it is relied on for the
axiomatisation of \logicKiF{} presented by van Ditmarsch, French and
Pinchinat~\cite{french2010future}. % TODO - include citation for cover operator

\begin{definition}[Semantics of \logicSF{}]
Let $M = (S, R, V)$ be a doxastic model. The interpretation of $\phi \in
\logicKDF$ is defined inductively.

\begin{eqnarray*}
M_s &\entails& p \text{ iff } s \in V_p\\
M_s &\entails& \neg \phi \text{ iff } M_s \nentails \phi\\
M_s &\entails& \phi \land \psi \text{ iff } M_s \entails \phi \text{ and } M_s
\entails \psi\\
M_s &\entails& \knows_a \phi \text{ if for all } t \in S : (s, t) \in R_a \text{
implies } M_t \entails \phi\\
M_s &\entails& \allrefs_a \phi \text{ iff for all } M'_{s'} \in \classS : M_s
\simulation_a M'_{s'} \text{ implies } M'_{s'} \entails \phi\\
\end{eqnarray*}
\end{definition}

The differences between \logicKF{} and \logicSF{} are similar to the differences
between \logicKF{} and \logicKDF{} discussed earlier. The class of models that
these logics are interpreted over are different, and particularly the
interpretation of the refinement operator means that \logicSF{} is not a
conservative extension of \logicKF{}. The example given earlier for \logicKDF{}
applies also to \logicSF{}: whilst $\somerefs_a \knows_a \bot$ is valid in
\logicKF{}, due to the reflexive nature of models in \classS{}, it is not valid
in \logicSF{}.

\begin{lemma}
The logic \logicSF{} is bisimulation invariant.
\end{lemma}

The proof for the bisimulation invariance of \logicKF{}, given by van Ditmarsch,
French and Pinchinat~\cite{french2010future} also applies to \logicKDF{}. 
% TODO - more explanation?

% TODO - examples

\section{Decision procedures}
