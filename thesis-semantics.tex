\chapter{Syntax and semantics}\label{semantics}

In this chapter we will introduce the syntax and semantics of the refinement
quantified modal logics, as introduced by van Ditmarsch and
French~\cite{french2009simulation}. The definitions that we give are general, in
the sense that the refinement quantified modal, doxastic and epistemic logics
are instances of these definitions.

We begin with a definition of the language, \langF{}, of the refinement
quantified modal logics, along with a definition of its semantics over a general
class of frames.

\begin{definition}[Language of \langF{}]
Given a finite set of agents $A$ and a set of propositional atoms $P$, the
language of \langF{} is defined by the following abstract syntax:
$$
\alpha ::=  p \bnfalt
            \neg \alpha \bnfalt
            \alpha \land \alpha \bnfalt
            \knows_a \alpha \bnfalt
            \allrefs_a \alpha
$$
where $p \in P$, $a \in A$ and $\alpha \in \lang{}$.
\end{definition}

We use the standard abbreviations from basic modal logic, along with an
abbreviation for the dual of the $\allrefs_a$ operator, $\somerefs_a \phi ::=
\neg \allrefs_a \neg \phi$.

We also use the cover operator $\covers_a \Gamma$, where $\Gamma$ is a finite
set of formulae, which is an abbreviation for 
$\covers_a \Gamma ::= \knows_a \bigvee_{\gamma \in \Gamma} \gamma \land
\bigwedge_{\gamma \in \Gamma} \suspects_a \gamma$. The cover operator is relied
on for our axiomatisation, in much the same way it is relied on for the
axiomatisation of \logicKiF{} presented by van Ditmarsch, French and
Pinchinat~\cite{french2010future}. % TODO - include citation for cover operator

We note that the basic modalities $\knows_a$ and $\suspects_a$ can be expressed
in terms of the cover operator, a fact that we will rely upon for all of our
axiomatisations. We note that $\knows_a \phi \iff \covers_a \{\phi\} \lor
\covers_a \emptyset$ and $\suspects_a \phi \iff \covers_a \{\phi, \top\}$. 
% TODO - citation?

\begin{definition}[Semantics of \logicCF{}]
Let \classC{} be a class of Kripke models, and let $M = (S, R, V) \in \classC$
be a Kripke model taken from \classC{}. The interpretation of $\phi$ is defined
inductively.
\begin{eqnarray*}
M_s &\entails& p \text{ iff } s \in V_p\\
M_s &\entails& \neg \phi \text{ iff } M_s \nentails \phi\\
M_s &\entails& \phi \land \psi \text{ iff } M_s \entails \phi \text{ and } M_s
\entails \psi\\
M_s &\entails& \knows_a \phi \text{ if for all } t \in S : (s, t) \in R_a \text{
implies } M_t \entails \phi\\
M_s &\entails& \allrefs_a \phi \text{ iff for all } M'_{s'} \in \classC : M_s
\simulation_a M'_{s'} \text{ implies } M'_{s'} \entails \phi
\end{eqnarray*}
\end{definition}

We use the same terminology and notation for satisfiability and validity as is
used for basic modal logics. As we sometimes discuss basic modal logics and
refinement quantified modal logics at the same time, we may add a subscript to
the turnstile operator, e.g. $\classK \entails_\somerefs \phi$, to make it
explicit that we are working in the refinement quantified version of the logic.

The logics \logicKF{}, \logicKDF{} and \logicSF{} are instances of \logicCF{},
with classes \classK{}, \classKD{} and \classS{} respectively. 

It should be emphasised that the interpretation of the refinement quantifier,
$\allrefs_a$, varies for each logic, as the refinements considered in the
interpretation of each logic must be taken from the appropriate class of Kripke
models. It is for this reason that \logicSF{} and \logicKDF{} are not
conservative extensions of \logicKF{}. For example, $\somerefs_a \knows_a \bot$
is valid in \logicKF{}, but not in \logicSF{} or \logicKDF{}. This is because
given any pointed model in \classK{}, one can construct an $a$-refinement from
that model by deleting the $a$-edges starting at the designated state; in this
resulting $a$-refinement, $\knows_a \bot$ is satisfied, and hence $\somerefs_a
\knows_a \bot$ is satisfied in the original model. However because of the serial
property of \classS{} and \classKD{} models, $\knows_a \bot$ is not even
satisfiable in \logicSF{} or \logicKDF{}, and hence $\somerefs_a \knows_a \bot$
is not satisfiable either.

We also note that for a class of frames \classC{}, the refinement quantified
modal logic \logicCF{} is a conservative extension of the basic modal logic
\logicC{}. This is because the interpretation of any formula that does not
contain a $\allrefs_a$ operator is the same as the interpretation of that
formula in basic modal logic.

\begin{lemma}
The logic \logicKF{} is bisimulation invariant.
\end{lemma}

This is proven by van Ditmarsch, French and Pinchinat~\cite{french2010future}.

\begin{lemma}
The logics \logicKDF{} and \logicSF{} are bisimulation invariant.
\end{lemma}

The proof for bisimulation invariance in \logicKF{}, given by van Ditmarsch,
French and Pinchinat~\cite{french2010future} applies to \logicSF{} and
\logicKDF{}.

% TODO - examples
\begin{proposition}[Properties of \logicKF{}, \logicKDF{} and \logicSF{}] Let
$\phi \in \langF$, let $\psi$ be an $a$-positive formula, and let $\pi$ be a
propositional formula. Then the following are valid:

\begin{enumerate}
\item $\classK \entails \allrefs_a \pi \iff \pi$
\item $\classK \entails \allrefs_a \phi \implies \phi$
\item $\classK \entails \allrefs_a \psi \iff \psi$
\item $\classK \entails \somerefs_a \knows_a \bot$
\item $\classK \entails \somerefs_a \somerefs_b \phi \iff \somerefs_b
\somerefs_a \phi$
\item $\classK \entails \somerefs_a \phi \iff \somerefs_a \somerefs_a \phi$
\item $\classS \entails \pi \iff \somerefs_a \knows_a \pi$
\end{enumerate}
\end{proposition}

% TODO - example where \neg \somerefs_a \phi and \phi is not just a
% contradictory statement in modal logic
