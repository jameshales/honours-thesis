\chapter{Single-agent refinement quantified epistemic and doxastic logics}\label{single}

In this chapter we will provide a sound and complete axiomatisation of the
single-agent refinement quantified epistemic and doxastic logics.  Our efforts
focus mainly on the epistemic variant; as we will see, the axiomatisation for
the doxastic variant is very similar, and can be derived using similar
reasoning. We do not go into much detail for the doxastic variant, as in
Chapter~\ref{kd45} we provide an axiomatisation for the multi-agent doxastic
variant, a much more general result.

When discussing the single-agent variants of these logics, we will use a
subscript $(1)$ to denote the single-agent logics, e.g. \logicSi{}, \logicSiF{},
and so on. We also drop the superfluous subscripts denoting agents in the syntax
of the logics, e.g. we write \knows{} instead of $\knows_a$.

\section{Technical preliminaries}

In previous work, van Ditmarsch, French and Pinchinat~\cite{french2010future}
gave an axiomatisation of the single-agent refinement quantified modal logic.
The axiomatisation was expressed in terms of the cover operator, $\covers$, and
the proof of completeness consisted of a translation from the refinement
quantified logic to the basic modal logic, a translation that relied on a
disjunctive normal form for modal formulae, defined in terms of the cover
operator. The cover operator allows us to collect all of the modalities that are
true at a particular state, and consider them all at once. This simplified the
soundness and completeness proofs of the modal refinement quantified logic. As
van Ditmarsch, French and Pinchinat~\cite{french2010future} remarked, a proof
directly in the setting of basic modal logic, without the cover operator, is
possible, but considerably longer.

Our axiomatisations of the single-agent epistemic and doxastic variants also
rely on the cover operator, and on a normal form similar to the disjunctive
normal form used by van Ditmarsch, French and Pinchinat~\cite{french2010future}.
The normal form that we introduce is called the cover logic prenex normal form,
which is similar to the disjunctive normal form, except that it prohibits nested
modalities; thus modalities may only be applied to propositional formulae. As we
will see, every single-agent epistemic and doxastic formula is equivalent to a
formula in this prenex normal form. The prenex normal form allows us to avoid
complications that would arise due to the transitive and symmetric properties of
\classS{}, or the transitive and Euclidean properties of \classKD{}, that would
arise when nested modalities are present.

We define the basic prenex normal form first, and then give the definition of
the cover logic version. We show that every single-agent epistemic or doxastic
formula is equivalent to a formula in these normal forms.

\begin{definition}[Prenex normal form]
A formula in prenex normal form is specified by the following abstract syntax: 
\begin{eqnarray*}
\alpha &::=& \delta \bnfalt \alpha \lor \alpha\\
\delta &::=&    \pi \bnfalt
                \knows \pi \bnfalt
                \suspects \pi \bnfalt
                \delta \land \delta
\end{eqnarray*}
where $\pi$ stands for a propositional formula.
\end{definition}

\begin{lemma}\label{single-prenex-s5}
Every formula of \langi{} is equivalent to a formula in prenex normal form, under
the semantics of \logicSi{}.
\end{lemma}

This is shown by Meyer and van der Hoek~\cite{meyer2004epistemic}.

\begin{lemma}\label{single-prenex-kd45}
Every formula of \langi{} is equivalent to a formula in prenex normal form, under
the semantics of \logicKDi{}.
\end{lemma}

\begin{proof}
The proof given by Meyer and van der Hoek~\cite{meyer2004epistemic} for
Lemma~\ref{single-prenex-s5} applies also to \logicKDi{}.

Meyer and van der Hoek remarked that the only use of the reflexivity axiom of
\logicS{}, {\bf T}, in the proof, is in the form of the theorems $\proves \knows
\knows \phi \implies \knows \phi$, and $\proves \knows \neg \knows \phi \implies
\neg \knows \phi$. Therefore the proof holds for any logic which replaces {\bf
T} with axioms entailing both of these properties. Both of these properties are
obviously valid in \logicKDi{}, and therefore the proof of
Lemma~\ref{single-prenex-s5} applies to this result.
\end{proof}

Given this result, we define our cover logic version of the prenex normal form,
and show that every epistemic or doxastic formula is equivalent to a formula in
cover logic prenex normal form.

\begin{definition}[Cover logic prenex normal form]
A formula in cover logic prenex normal form is specified by the following
abstract syntax: 
$$\alpha ::= \pi \land \covers \Gamma \bnfalt \alpha \lor \alpha$$
where $\pi$ is a propositional formula, and $\Gamma$ is a set of propositional
formulae.
\end{definition}

\begin{lemma}\label{single-cover-prenex}
Every formula of \langi{} is equivalent to a formula in cover logic prenex normal
form, under both the semantics of \logicSi{} and \logicKDi{}.
\end{lemma}

\begin{proof}
Without loss of generality, we may assume that our given formula is in prenex
normal form (by Lemma~\ref{single-prenex-s5} for \logicSi{}, or by
Lemma~\ref{single-prenex-kd45} for \logicKDi{}).

Given a formula in prenex normal form, we consider each disjunct separately. We
can convert each term $\knows \gamma$ or $\suspects \gamma$ into an equivalent
term using the cover operator, using the equivalences $\knows \gamma \equiv
\covers \{ \gamma \}$ and $\suspects \gamma \equiv \covers \{ \gamma, \top \}$ 
Note that each resulting term contains a cover operator applied only to
a set of propositional formulae.

An inductive argument can be used to show that we can collapse the resulting
conjunction of cover operators into a single term containing one cover
operator applied to a set of propositional formulae. We use the following
equivalence to achieve this, and note that at each stage this equivalence
preserves the property that the cover operator is only applied to a set of
propositional formulae.
$$
\covers \Gamma \land \covers \Gamma' \equiv 
\covers \big( 
\{ \gamma \land \bigvee_{\gamma' \in \Gamma'} \gamma' \mid \gamma \in \Gamma \}
\cup
\{ \gamma' \land \bigvee_{\gamma \in \Gamma} \gamma \mid \gamma' \in \Gamma' \}
\big)
$$

Repeating this for each disjunct in our original formula leaves us with a
formula in cover logic prenex normal form.
\end{proof}

\section{Axiomatisation}

We provide an axiomatisation of the single-agent refinement quantified epistemic
and doxastic logics, \logicSiF{} and \logicKDiF{}, and prove their soundness
and completeness.

As we are defining several similar axiomatisations at once, we will define their
common elements as \axiomFi{}, and define the axiomatisations of \logicKDiF{}
and \logicSiF{}, \axiomKDiF{} and \axiomSiF{} respectively, as extensions of
\axiomFi{}. We will also define the axiomatisation, \axiomKiF{} of single-agent
refinement quantified modal logic, so that we can compare our axiomatisations.

\begin{definition}
The axiomatisation \axiomFi{} is a substitution schema of the following axioms:
$$
\begin{array}{rl}
{\bf P} & \text{All tautologies of propositional logic.}\\
{\bf K} & \knows (\phi \implies \psi) \implies \knows \phi \implies \knows \psi\\
{\bf R} & \allrefs (\phi \implies \psi) \implies \allrefs \phi \implies \allrefs \psi\\
{\bf RP} & \allrefs \alpha \iff \alpha \text{, where $\alpha$ is a propositional formula.}\\
\end{array}
$$

Along with the rules:
$$
\begin{array}{rl}
{\bf MP} & \text{From $\proves \phi \implies \psi$ and $\proves \phi$ infer
$\proves \psi$.}\\
{\bf NecK} & \text{From $\proves \phi$ infer $\proves \knows \phi$.}\\
{\bf NecR} & \text{From $\proves \phi$ infer $\proves \allrefs \phi$.}
\end{array}
$$
\end{definition}

The axioms and rules in \axiomFi{} were previously presented by van Ditmarsch,
French and Pinchinat as axioms of \axiomKiF{}, which we will define now.

\begin{definition}
The axiomatisation \axiomKiF{} is a substitution schema consisting of the axioms and rules of \axiomFi{}, along with the additional axiom:
$$
\begin{array}{rl}
{\bf RK} & \somerefs \covers \Gamma \iff \bigwedge_{\gamma \in
\Gamma} \suspects \somerefs \gamma
\end{array}
$$
\end{definition}

The axiomatisation \axiomKiF{} shares axioms and rules from the axiomatisation
for basic modal logic, \axiomK{}. 

\begin{definition}
The axiomatisation \axiomSiF{} is a substitution schema consisting of the axioms
and rules of \axiomFi{}, along with the additional axioms:
$$
\begin{array}{rl}
{\bf T} & \knows \phi \implies \phi\\
{\bf 5} & \suspects \phi \implies \knows \suspects \phi\\
{\bf RS5} & \somerefs \covers \Gamma \iff \bigvee_{\gamma \in \Gamma} \gamma \land 
\bigwedge_{\gamma \in \Gamma} \suspects \gamma
\end{array}
$$
where $\Gamma$ is a set of propositional formulae.
\end{definition}

\begin{definition}
The axiomatisation \axiomKDiF{} is a substitution schema consisting of the axioms
and rules of \axiomFi{}, along with the additional axioms:
$$
\begin{array}{rl}
{\bf D} & \knows \phi \implies \suspects \phi\\
{\bf 4} & \knows \phi \implies \knows \knows \phi\\
{\bf 5} & \suspects \phi \implies \knows \suspects \phi\\
{\bf RKD45} & \somerefs \covers \Gamma \iff \bigwedge_{\gamma \in \Gamma} \suspects \gamma
\end{array}
$$
where $\Gamma$ is a set of propositional formulae.
\end{definition}

We note that many of the axioms from \axiomSiF{} and \axiomKDiF{} are also
axioms from \axiomS{}, \axiomKD{} and \axiomKiF{}. The differences between
these two axiomatisations, and \axiomKiF{}, is that \axiomSiF{} and \axiomKDiF{}
include the additional \axiomS{} and \axiomKD{} axioms respectively, and include
a different axiom in place of {\bf RK}.

\begin{example}
We give an example derivation using \axiomSiF{}, showing that
$\proves_{\axiomSiF} p \implies \somerefs \knows p$.
$$
\begin{array}{ll}
\proves \knows \neg p \implies \neg p & \text{({\bf T})}\\
\proves p \implies p \land \neg \knows \neg p & \text{({\bf P})}\\
\proves p \implies p \land \suspects p & \text{(Definition of $\suspects$)}\\
\proves p \implies p \land \suspects \neg \neg p & \text{({\bf P})}\\
\proves p \implies p \land \suspects \neg \allrefs \neg p & \text{({\bf RP})}\\
\proves p \implies p \land \suspects \somerefs p & \text{(Definition of $\somerefs$)}\\
\proves p \implies \somerefs \covers \{ p \} & \text{({\bf RS5})}\\
\proves p \implies \somerefs \knows p & \text{(Definition of $\covers$)}
\end{array}
$$
\end{example}

\begin{example}
We give an example derivation using \axiomKDiF{}, showing that
$\proves_{\axiomKDiF} \suspects p \implies \somerefs \knows p$.
$$
\begin{array}{ll}
\proves \suspects p \implies \suspects p & \text{({\bf P})}\\
\proves \suspects p \implies \suspects \neg \neg p & \text{({\bf P})}\\
\proves \suspects p \implies \suspects \neg \allrefs \neg p & \text{({\bf RP})}\\
\proves \suspects p \implies \suspects \somerefs p & \text{(Definition of $\somerefs$)}\\
\proves \suspects p \implies \somerefs \covers \{ p \} & \text{({\bf RKD45})}\\
\proves \suspects p \implies \somerefs \knows p & \text{(Definition of $\covers$)}
\end{array}
$$
\end{example}

We will now show the soundness of the axiomatisations \axiomSiF{} and
\axiomKDiF{}.

\begin{lemma}
The axiomatisation \axiomFi{} is sound with respect to the semantic classes of
\classKi{}, \classSi{} and \classKDi{}.
\end{lemma}

\begin{proof}
The soundness of the axioms {\bf P} and {\bf K}, and the rules {\bf MP} and {\bf
NecK} can be shown by the same reasoning used to show that they are sound in
basic modal logic. The axioms {\bf R} and {\bf RP}, and the rule {\bf NecR} can
be shown to be sound using the same reasoning applied in the single-agent
refinement quantified modal logic~\cite{french2010future}.
\end{proof}

\begin{lemma}\label{single-sound-s5}
The axiomatisation \axiomSiF{} is sound for the logic \logicSiF{}.
\end{lemma}

\begin{proof}
Soundness of the axioms {\bf P}, {\bf K}, {\bf R}, and {\bf RP}, and the rules
{\bf MP}, {\bf NecK} and {\bf NecR} are shown above. Soundness of the axioms
{\bf T} and {\bf 5} can be shown by the same reasoning used to show that they
are sound in basic epistemic logic.

All that is to be shown is the soundness of {\bf RS5}.

Let $\Gamma$ be a finite set of propositional formulae, and let $M_s$ be a model in
\classS{} such that $M_s \entails \bigvee_{\gamma \in \Gamma} \gamma \land
\bigwedge_{\gamma \in \Gamma} \suspects \gamma$.

We need to show that $M_s \entails \somerefs \covers \Gamma$. To do this we will
construct a model $N_s \in \classS$, construct a simulation from $N_s$ to $M_s$
to show that $N_s \refinement M_s$, and finally show that $N_s \entails
\covers \Gamma$.

We begin by constructing the model $N_s$. We note that for each $\gamma \in
\Gamma$, there is some successor $s^\gamma \in sR^{M}$ such that $M_{s^\gamma}
\entails \gamma$.  We can construct the model $N$ such that $S^{N} = \{s\} \cup
\{s^\gamma \mid \gamma \in \Gamma\}$, $R^{N} = S^{N} \times S^{N}$ and for all
$p \in P$, $V^{N}(p) = V^M(p) \cap S^{N}$. This model is clearly in \classS.

Furthermore we have that $M'_s \refinement M_s$ by the relation ${\cal
R} = \{(s, s)\} \cup \{(t^\gamma, t^\gamma) \mid \gamma \in \Gamma\}$
({\bf atoms} and {\bf forth} are satisfied). 

By construction, for each $\gamma \in \Gamma$, there is a successor $s^\gamma
\in sR^{N}$ such that $N_{s^\gamma} \entails \gamma$. For each successor $s
\in sR^{N}$ we have that $N_s \entails \bigvee_{\gamma \in \Gamma}
\gamma$, as each successor is either one of the $s^\gamma$ for some $\gamma \in
\Gamma$, in which case $N_{s^\gamma} \entails \gamma$, or it is our initial
state $s$, in which case $N_s \entails \bigvee_{\gamma \in \Gamma} \gamma$
follows from our hypothesis that $M_s \entails \bigvee_{\gamma \in \Gamma}
\gamma$. Therefore $N_s \entails \covers \Gamma$.

Therefore $M_s \entails \somerefs \covers \Gamma$. 

Conversely, let $\Gamma$ be a finite set of propositional formulae, and let
$M_s$ be a model in \classS{} such that $M_s \entails \somerefs \covers \Gamma$.
Then there exists some model $N_t \refinement M_s$ in \classS, via some
simulation ${\cal R} \subseteq S' \times S$, such that $N_t \entails \covers
\Gamma$.

From the definition of the cover operator, $N_t \entails \knows
\bigvee_{\gamma \in \Gamma} \gamma \land \bigwedge_{\gamma \in \Gamma} \suspects
\gamma$. 

As $N \in \classS$, we know that $t \in tR^N$, and so it follows from
$N_t \entails \knows \bigvee_{\gamma \in \Gamma} \gamma$ that $N_t
\entails \bigvee_{\gamma \in \Gamma} \gamma$. As we know that $(t, s) \in {\cal
R}$, from {\bf atoms} we know that $M_s$ and $N_t$ are equivalent for
propositional formulae. As each $\gamma \in \Gamma$ is propositional, it follows
that $M_s \entails \bigvee_{\gamma \in \Gamma} \gamma$.

Furthermore, from $N_t \entails \bigwedge_{\gamma \in \Gamma} \suspects
\gamma$, we know that for every $\gamma \in \Gamma$, there exists some
$t^\gamma \in tR^{M'}$ such that $N_{t^\gamma} \entails \gamma$. It then
follows from {\bf forth} that there exists some $s^\gamma \in S^M$ such that
$s^\gamma \in sR^M$ and $(t^\gamma, s^\gamma) \in {\cal R}$. From {\bf atoms}
we know that $M_{s^\gamma}$ and $N_{t^\gamma}$ are equivalent for
propositional formulae. As $\gamma$ is propositional, it follows that
$M_{s^\gamma} \entails \gamma$ and therefore $M_s \entails \suspects \gamma$ for
each $\gamma \in \Gamma$.  Therefore $M_s \entails \bigwedge_{\gamma \in \Gamma}
\suspects \gamma$, and so $M_s \entails \bigvee_{\gamma \in \Gamma} \gamma \land
\bigwedge_{\gamma \in \Gamma} \suspects \gamma$.

Therefore {\bf RS5} is sound, and so \axiomSiF{} is sound for the logic
\logicSiF{}.
\end{proof}

\begin{lemma}\label{single-sound-kd45}
The axiomatisation \axiomKDiF{} is sound with respect to the semantic class
\classKDi{}.
\end{lemma}

\begin{proof}
The proof is similar to the proof for Lemma~\ref{single-sound-s5}. Instead of
showing soundness for the \axiomSi{} axioms, we must show that the \axiomKDi{}
axioms are sound, and this follows from their soundness in \logicKDi{}. The main
difference in the proof of soundness is for {\bf RKD45} as compared to the proof
for {\bf RS5}, is that in the right to left direction, we do not have to show
that $M_s \entails \bigvee_{\gamma \in \Gamma} \gamma$; as doxastic models do
not have to be reflexive, there is no requirement for $s$ to be in the possible
worlds for the constructed refinement. For the left to right direction of the
proof, the refinement $N_t$ is a KD45 model instead of an S5 model, but this has
no bearing on the rest of the proof.
\end{proof}

We show the completeness of the axiomatisations \axiomSiF{} and \axiomKDiF{} by
provably correct translations from \langFi{} to \langi{}, using the axioms and
rules of \axiomSiF{} and \axiomKDiF{}. A provably correct translation uses the
axioms of our respective axiomatisations in order to translate our given formula
to an equivalent formula in a different form. In our case, we are translating
our formulae into a form that does not include any refinement quantifiers (i.e.
\somerefs{}-free formulae). We can show from the interpretation of refinement
quantified modal logic that the interpretation of a \somerefs{}-free formula is
equivalent to the interpretation of that same formula in basic modal logic, and
that therefore we can prove theorems in the refinement quantified modal logic,
by translating the formula to a \somerefs{}-free form and then proving that
theorem in basic modal logic. The completeness of our axiomatisations then
follows from the completeness of the respective basic modal logics. 

This is the same strategy used by van Ditmarsch, French and
Pinchinat~\cite{french2010future} to show the completeness of the single-agent
refinement quantified modal logic, and is the same general strategy that we will
use for the other logics that we consider in this paper. We will begin by
providing a provably correct translation from \langFi{} to \langi{}, and follow
with some results that show that completeness follows from this translation.

\begin{lemma}\label{single-reduction-s5}
Every formula of \langFi{} is provably equivalent to a formula of \langi{} with
the axiomatisation \axiomSiF{}.
\end{lemma}

\begin{proof}
Let $\alpha \in \langFi{}$. We assume without loss of generality that all
\allrefs{} operators are expressed as \somerefs{} operators, by the equivalence
$\allrefs \phi \iff \neg \somerefs \neg \phi$. We prove by induction on the
number of occurrences of \somerefs{} in $\alpha$ that $\alpha$ is equivalent to
a \somerefs{}-free formula, and therefore to a formula in \langi{}. The base case
where $\alpha$ contains no \somerefs{} is trivial, as a \somerefs{}-free formula
is a formula in \langi{}. Suppose instead that $\alpha$ contains $n + 1$
\somerefs{} operators, and assume that any formula with $n$ \somerefs{}
operators is provably equivalent to a formula in \langi{}. We use the axioms of
\axiomSiF{} to show that $\alpha$ is provably equivalent to a formula with $n$
\somerefs{} operators, and that therefore by the induction hypothesis it is
provably equivalent to a formula in \langi{}.

Given any subformula from $\alpha$ of type $\somerefs \beta$, such that $\beta$
is a \somerefs{}-free formula. Without loss of generality, by
Lemma~\ref{single-cover-prenex} we may assume that $\beta$ is in disjunctive
normal form. We prove by induction on the structure of $\beta$ that $\somerefs
\beta$ is provably equivalent to a formula $\chi \in \langi$. The induction
hypothesis is that for any proper subformula $\phi$ of $\beta$ that $\somerefs
\phi$ is equivalent to a formula in \langi{}.

The base case is when $\beta$ is a propositional formula. In this case, from
{\bf P} and {\bf RP}, we have that $\somerefs \beta \iff \beta$, and therefore
$\somerefs \beta$ is equivalent to a formula in \langi{}.

The inductive case is when $\beta = \phi \lor \psi$, or when $\beta = \pi \land
\covers \Gamma$. 

Suppose that $\beta = \phi \lor \psi$. Then $\somerefs (\phi \lor \psi) \iff
\somerefs \phi \lor \somerefs \psi$ is derivable from {\bf P} and {\bf R}. By
the induction hypothesis, $\somerefs \phi$ and $\somerefs \psi$ are equivalent
to some $\phi', \psi' \in \langi{}$. Therefore $\somerefs \beta \iff \phi' \lor
\psi'$, and so $\somerefs \beta$ is equivalent to a formula in \langi{}.

Suppose that $\beta = \pi \land \covers \Gamma$ where $\Gamma$ is a set of
propositional formulae. Then $\somerefs (\pi \land \covers \Gamma) \iff
\pi \land \somerefs \covers \Gamma$ is derivable from {\bf P}, {\bf R} and {\bf
RP}. Moreover, $\pi \land \somerefs \covers \Gamma \iff \pi \land
\bigvee_{\gamma \in \Gamma} \gamma \land \bigwedge_{\gamma \in \Gamma} \suspects
\gamma$ is derivable from {\bf RS5}. As each $\gamma \in \Gamma$ is
propositional, $\somerefs \beta$ is equivalent to a formula in \langi{}.

Therefore by the principle of mathematical induction, $\somerefs \beta$ is
equivalent to a formula $\chi \in \langi{}$ for every $\beta \in \langFi{}$.

Hence replacing $\somerefs \beta$ in $\alpha$ with $\chi$ gives an equivalent
formula that contains only $n$ \somerefs{} operators.

Therefore by the principle of mathematical induction, $\alpha$ is equivalent to
a formula in \langi{}.
\end{proof}

\begin{lemma}\label{single-reduction-kd45}
Every formula of \langFi{} is provably equivalent to a formula of \langi{} with
the axiomatisation \axiomKDiF{}.
\end{lemma}

The proof is similar to the proof for Lemma~\ref{single-reduction-s5}, with the
only difference being that the axiom {\bf RKD45} is used in place of {\bf RS5}
in the induction over the structure of $\beta$.

The rest of the completeness proof is to show that, given the above translations
into \langi{}, we can show completeness by using these translations along with
the completeness of \logicSi{} and \logicKDi{}.

\begin{corollary}\label{single-derivable-s5}
Let $\phi \in \langFi$ be given and $\psi \in \langi$ be equivalent to $\phi$
under the semantics of \logicSiF{}.  If $\psi$ is a theorem in \logicSi{}, then
$\phi$ is a theorem in \axiomSiF{}.
\end{corollary}

\begin{proof}
Let $\phi \in \langFi$ and let $\psi \in \langi$ be semantically equivalent to
$\phi$. By Lemma~\ref{single-reduction-s5}, we can obtain some $\phi' \in \langi$
that is semantically equivalent to $\phi$ (and thus also to $\psi$) by following
the given translation steps. We can extend a derivation of $\psi$ to a
derivation of $\phi'$ as the two are semantically equivalent under \logicSi{}, and by
the completeness of \logicSi{} this equivalence is derivable. As \axiomSiF{} is a
conservative extension of \logicSi{}, this equivalence is therefore also derivable
in \axiomSiF{}. The derivation can be further extended to $\phi$ by observing that all
of the reduction steps in Lemma~\ref{single-reduction-s5} are provable equivalences
in \axiomSiF{}. Therefore $\phi$ is a theorem in \axiomSiF{}.
\end{proof}

\begin{lemma}\label{single-complete-s5}
The axiom schema \axiomSiF{} is complete for the logic \logicSiF{}.
\end{lemma}

\begin{proof}
Let $\phi \in \langFi$ such that $\entails_\somerefs \phi$. Then by
Lemma~\ref{single-reduction-s5}, there exists a semantically equivalent formula
$\psi \in \langi$ which is \somerefs-free. As $\entails_\somerefs \phi$ and
$\phi \iff \psi$, then $\entails_\somerefs \psi$. As $\psi$ is
\somerefs-free, then it follows that $\entails \psi$, and by the
completeness of \axiomSiF{} it follows that $\proves \psi$.
Therefore by Corollary~\ref{single-derivable-s5} we have that $\proves_\somerefs
\phi$.
\end{proof}

\begin{theorem}
The axiomatisation \axiomSiF{} is sound and complete with respect to the
semantic class \classSi{}.
\end{theorem}

\begin{proof}
The soundness proof is given in Lemma~\ref{single-sound-s5} and the completeness
proof is given in Lemma~\ref{single-complete-s5}.
\end{proof}

\begin{theorem}
The axiomatisation \axiomKDiF{} is sound and complete with respect to the
semantic class \classKDi{}.
\end{theorem}

\begin{proof}
The soundness proof is given in Lemma~\ref{single-sound-kd45}, and we note that
similar results to Corollary~\ref{single-derivable-s5} and
Lemma~\ref{single-complete-s5} can be shown with minor modifications to their
proofs, which gives us completeness.
\end{proof}

The completeness proofs above were performed with a provably correct translation
from \langFi{} to \langi{}, under the semantics of \logicSiF{} and \logicKDiF{}.
This shows that \logicSiF{} and \logicKDiF{} are expressively equivalent to
\logicSi{} and \logicKDi{} respectively. This allows us to show that \logicSiF{}
and \logicKDiF{} share properties of \logicSi{} and \logicKDi{}, via this
property. In particular, we will show that \logicSiF{} and \logicKDiF{} are
decidable; that is, that there is a decision procedure for determining whether
any formula in \langFi{} is valid in \logicSiF{} or \logicKDiF{}.

% TODO - expressivity ???

\begin{theorem}\label{single-decidable}
The logics \logicSiF{} and \logicKDiF{} are decidable.
\end{theorem}

\begin{proof}
Given a formula $\phi$ in \logicSiF{}, we can find an equivalent $\psi$ in
\logicSi{} (from Lemma~\ref{single-reduction-s5}). We can therefore determine whether
$\psi$ is satisfiable using a decision procedure designed for \logicSi{}. The
decidability for \logicSiF{} therefore follows from the decidability of
\logicSi{}~\cite{blackburn2002modal}.

The proof for \logicKDiF{} is the same, but relying on
Lemma~\ref{single-reduction-kd45} for the translation, and on the decidability of
\logicKDi{}~\cite{blackburn2002modal}.
\end{proof}

We note that other results from single-agent epistemic and doxastic logics can
be shown in the setting of the refinement quantified versions by using similar
reasoning.
