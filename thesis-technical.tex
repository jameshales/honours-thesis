\chapter{Technical preliminaries}

We recall the definitions given by van Ditmarsch, French and
Pinchinat~\cite{french2010future} in describing the future event logic, and we
adapt those definitions to be suitable for doxastic logic (\logicKD{}) and
epistemic logic (\logicS{}).  Specifically, when we are discussing the
extensions of the future event logic to doxastic and epistemic logics, we
restrict the Kripke models under discussion to those in the class of doxastic or
epistemic models respectively.

Let $A$ be a non-empty, finite set of agents, and let $P$ be a non-empty,
countable set of propositional atoms.

\begin{definition}[Kripke model]
A \textit{Kripke model} $M = (S, R, V)$ consists of a \textit{domain} $S$, which
is a set of states (or worlds), \textit{accessibility} $R : A \to \mathcal{P}(S
\times S)$, and a \textit{valuation} $V : P \to \mathcal{P}(S)$. 

The class of all Kripke models is called \classK{}. We write $M \in \classK{}$
to denote that $M$ is a Kripke model.
\end{definition}

For $R(a)$, we write $R_a$. We write $sR_a$ for $\{t \mid (s, t) \in R_a\}$ and we
write $R_at$ for $\{s \mid (s, t) \in R_a\}$. As we will be required to discuss
several models at once, we will use the convention that $M = (S^M, R^M, V^M)$,
$N = (S^N, R^N, V^N)$, and so on. For $s \in S^M$ we will let $M_s$ refer to the
pair $(M, s)$, also known as the pointed Kripke model of $M$ at state $s$.

\begin{definition}[Doxastic model]
A \textit{doxastic model} is a Kripke model $M = (S, R, V)$ such that the
relation $R_a$ is serial, transitive, and Euclidean for all $a \in A$. The class
of all doxastic models is called \classKD{}. We write $M \in \classKD{}$ to
denote that $M$ is a doxastic model.
\end{definition}

\begin{definition}[Epistemic model]
An \textit{epistemic model} is a Kripke model $M = (S, R, V)$ such that the
relation $R_a$ is an equivalence relation for all $a \in A$. The class of all
epistemic models is called \classS{}. We write $M \in \classS{}$ to denote that
$M$ is an epistemic model.
\end{definition}

Throughout this paper we will be presenting results in both doxastic and
epistemic logic.  As such, when we are discussing doxastic logic, we will
assume that all Kripke models are implicitly doxastic models, and likewise when
we are discussing epistemic logic, we will assume that all Kripke models are
implicitly epistemic models, unless this has to be explicitly proven as part of
a proof. When we are discussing results in general modal logic, we will not
assume any restrictions on the Kripke models.

\begin{definition}[Bisimulation]
Let $M = (S, R, V)$ and $M' = (S', R', V')$ be Kripke models. A non-empty
relation $\mathcal{R} \subseteq S \times S'$ is a \textit{bisimulation} if and
only if for all $s \in S$ and $s' \in S'$, with $(s, s') \in \mathcal{R}$, for
all $a \in A$:

\begin{description}
\item[atoms] $s \in V(p)$ if and only if $s' \in V'(p)$ for all
$p \in P$

\item[forth-$a$] for all $t \in S$, if $R_a(s, t)$, then there is a
$t' \in S'$ such that $R'_a(s', t')$ and $(t,
t') \in \mathcal{R}$

\item[back-$a$] for all $t' \in S'$, if $R'_a(s',
t')$, then there is a $t \in S$ such that $R_a(s, t)$ and $(t, t')
\in \mathcal{R}$.
\end{description}

We call $M_s$ and $M'_{s'}$ bisimilar, and write $M_s \bisim M'_{s'}$ to denote
that there is a bisimulation between $M_s$ and $M'_{s'}$.
\end{definition}

\begin{definition}[Simulation and refinement]
Let $M$ and $M'$ be Kripke models. A non-empty relation $\mathcal{R}
\subseteq S \times S'$ is a \textit{simulation} if and only if it satisfied {\bf
atoms} and {\bf forth-$a$} for every $a \in A$.

We call $M'_{s'}$ a simulation of $M_s$ and we call $M_s$ a refinement of
$M'_{s'}$. We write $M'_{s'} \simulation M_s$ to denote this, or alternatively,
$M_s \refinement M'_{s'}$.

A relation that satisfies {\bf atoms} and {\bf forth-$b$} for every $b \in A$,
and satisfies {\bf back-$b$} for every $b \in A - \{a\}$, for some $a \in A$, is
an $a$\textit{-simulation}. 

We call $M'_{s'}$ an $a$-simulation of $M_s$, and we call $M_s$ an
$a$-refinement of $M'_{s'}$. We write $M'_{s'} \simulation_a M_s$ to denote
this, or alternatively, $M_s \refinement_a M'_{s'}$.
\end{definition}

Whilst we have introduced simulations and refinements as relations between
models, we will often also refer to Kripke models as refinements or simulations
of other models. For example, if $M_s \refinement_a M'_{s'}$ then we call $M_s$
an $a$-refinement of $M'_{s'}$, and we call $M'_{s'}$ an $a$-simulation of
$M_s$.

We will use $a$-refinements to define the semantics of the future event logic.
The significance of refinements is that the finite refinements of a Kripke model
are exactly the models that result from the execution of an action
model~\cite{french2009simulation}.
