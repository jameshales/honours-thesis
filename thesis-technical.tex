\chapter{Technical preliminaries}

%We recall the definitions given by van Ditmarsch, French and
%Pinchinat~\cite{french2010future} in describing the future event logic, and we
%adapt those definitions to be suitable for doxastic logic (\logicKD{}) and
%epistemic logic (\logicS{}).  Specifically, when we are discussing the
%extensions of the future event logic to doxastic and epistemic logics, we
%restrict the Kripke models under discussion to those in the class of doxastic or
%epistemic models respectively.

% TODO - Pre-amble about modal logics

Let $A$ be a non-empty, finite set of agents, and let $P$ be a non-empty,
countable set of propositional atoms.

\begin{definition}[Kripke model]
A \textit{Kripke model} $M = (S, R, V)$ consists of a \textit{domain} $S$, which
is a set of states (or worlds), \textit{accessibility} $R : A \to \mathcal{P}(S
\times S)$, and a \textit{valuation} $V : P \to \mathcal{P}(S)$. 

The class of all Kripke models is called \classK{}. We write $M \in \classK{}$
to denote that $M$ is a Kripke model.
\end{definition}

For $R(a)$, we write $R_a$. We write $sR_a$ for $\{t \mid (s, t) \in R_a\}$ and
we write $R_at$ for $\{s \mid (s, t) \in R_a\}$. As we will be required to
discuss several models at once, we will use the convention that $M = (S^M, R^M,
V^M)$, $N = (S^N, R^N, V^N)$, and so on. For $s \in S^M$ we will let $M_s$ refer
to the pair $(M, s)$, also known as the pointed Kripke model of $M$ at state
$s$.

\begin{definition}[Doxastic model]
A \textit{doxastic model} is a Kripke model $M = (S, R, V)$ such that the
relation $R_a$ is serial, transitive, and Euclidean for all $a \in A$. The class
of all doxastic models is called \classKD{}. We write $M \in \classKD{}$ to
denote that $M$ is a doxastic model.
\end{definition}

\begin{definition}[Epistemic model]
An \textit{epistemic model} is a Kripke model $M = (S, R, V)$ such that the
relation $R_a$ is an equivalence relation for all $a \in A$. The class of all
epistemic models is called \classS{}. We write $M \in \classS{}$ to denote that
$M$ is an epistemic model.
\end{definition}

% TODO - Motivation

\begin{definition}[Language of \lang{}]
Given a finite set of agents $A$ and a set of propositional atoms $P$, the
language of \langF{} is defined by the following abstract syntax:
$$
\phi ::=    p \bnfalt
            \neg \phi \bnfalt
            \phi \land \phi \bnfalt
            \knows_a \phi \bnfalt
$$
where $p \in P$.
\end{definition}

Standard abbreviations include:
$\top ::= \phi \lor \neg \phi$;
$\bot ::= \neg \top$;
$\phi \lor \psi ::= \neg (\neg \phi \land \neg \psi)$;
$\phi \implies \psi ::= \neg \phi \lor \psi$;
and $\suspects_a \phi ::= \neg \knows_a \neg \phi$.

% TODO - Motiviation - \knows can be thought of as...

\begin{definition}[Semantics of \logicK{}]
Let \classC{} be a class of Kripke models, and let $M = (S, R, V) \in \classC$
be a Kripke model taken from \classC{}. The interpretation of $\phi \in \logicK$
is defined inductively.
\begin{eqnarray*}
M_s &\entails& p \text{ iff } s \in V_p\\
M_s &\entails& \neg \phi \text{ iff } M_s \nentails \phi\\
M_s &\entails& \phi \land \psi \text{ iff } M_s \entails \phi \text{ and } M_s
\entails \psi\\
M_s &\entails& \knows_a \phi \text{ if for all } t \in S : (s, t) \in R_a \text{
implies } M_t \entails \phi\\
\end{eqnarray*}
\end{definition}

The logics \logicK{}, \logicKD{} and \logicS{} are instances of \logicC{} with
classes \classK{}, \classKD{} and \classS{} respectively. It should be noted
that \logicKD{} is a conservative extension of \logicK{}, and \logicS{} is a
conservative extension of \logicKD{} (and also of \logicK{}). This is because
any formula that is valid with respect to a particular class of Kripke models
is also valid for any subset of those Kripke models.

% TODO - Properties of K, KD45, S5
% TODO - Examples

\begin{definition}[Axiomatisation \axiomK{}]
The axiomatisation \axiomK{} is a substitution schema consisting of the
following axioms:
$$
\begin{array}{rl}
{\bf P} & \text{All propositional tautologies}\\
{\bf K} & \knows (\phi \implies \psi) \implies \knows \phi \implies \knows
\psi\\
\end{array}
$$
Along with the rules:
$$
\begin{array}{rl}
{\bf MP} & \text{From $\proves \phi \implies \psi$ and $\proves \phi$, infer
$\proves \psi$}\\
{\bf NecK} & \text{From $\proves \phi$ infer $\proves \knows_a \phi$}\\
\end{array}
$$
\end{definition}

% TODO - Find citation

\begin{definition}[Axiomatisation \axiomKD{}]
The axiomatisation \axiomKD{} is a substitution schema consisting of the axioms
and rules of \axiomK{}, along with the additional axioms:
$$
\begin{array}{rl}
{\bf D} & \knows \phi \implies \suspects \phi\\
{\bf 4} & \knows \phi \implies \knows \knows \phi\\
{\bf 5} & \suspects \phi \implies \knows \suspects \phi\\
\end{array}
$$
\end{definition}

The additional axioms for \axiomKD{} correspond to the properties of \classKD{}
models. {\bf D} corresponds to the serial property, {\bf 4} corresponds to the
transitive property, and {\bf 5} corresponds to the Euclidean property.

% TODO - Find citation

\begin{definition}[Axiomatisation \axiomS{}]
The axiomatisation \axiomS{} is a substitution schema consisting of the axioms
and rules of \axiomK{}, along with the additional axioms:
$$
\begin{array}{rl}
{\bf T} & \knows \phi \implies \suspects \phi\\
{\bf 5} & \suspects \phi \implies \knows \suspects \phi\\
\end{array}
$$
\end{definition}

Once again, the additional axioms for \axiomS{} correspond to the properties of
\classS{} models. {\bf T} corresponds to the reflexive property, and {\bf 5}
corresponds to the Euclidean property, noting that reflexivity and Euclideaness
together imply transitivity and symmetry.

% TODO - Find citation

Throughout this paper we will be presenting results in both doxastic and
epistemic logic.  As such, when we are discussing doxastic logic, we will
assume that all Kripke models are implicitly doxastic models, and likewise when
we are discussing epistemic logic, we will assume that all Kripke models are
implicitly epistemic models, unless this has to be explicitly proven as part of
a proof. When we are discussing results in general modal logic, we will not
assume any restrictions on the Kripke models.

\begin{definition}[Bisimulation]
Let $M = (S, R, V)$ and $M' = (S', R', V')$ be Kripke models. A non-empty
relation $\mathcal{R} \subseteq S \times S'$ is a \textit{bisimulation} if and
only if for all $s \in S$ and $s' \in S'$, with $(s, s') \in \mathcal{R}$, for
all $a \in A$:

\begin{description}
\item[atoms] $s \in V(p)$ if and only if $s' \in V'(p)$ for all
$p \in P$

\item[forth-$a$] for all $t \in S$, if $R_a(s, t)$, then there is a
$t' \in S'$ such that $R'_a(s', t')$ and $(t,
t') \in \mathcal{R}$

\item[back-$a$] for all $t' \in S'$, if $R'_a(s',
t')$, then there is a $t \in S$ such that $R_a(s, t)$ and $(t, t')
\in \mathcal{R}$.
\end{description}

We call $M_s$ and $M'_{s'}$ bisimilar, and write $M_s \bisim M'_{s'}$ to denote
that there is a bisimulation between $M_s$ and $M'_{s'}$.
\end{definition}

\begin{lemma}
The logics \logicK{}, \logicKD{} and \logicS{} are bisimulation invariant.
\end{lemma}

% TODO - Find citation

\begin{definition}[Simulation and refinement]
Let $M$ and $M'$ be Kripke models. A non-empty relation $\mathcal{R}
\subseteq S \times S'$ is a \textit{simulation} if and only if it satisfied {\bf
atoms} and {\bf forth-$a$} for every $a \in A$.

We call $M'_{s'}$ a simulation of $M_s$ and we call $M_s$ a refinement of
$M'_{s'}$. We write $M'_{s'} \simulation M_s$ to denote this, or alternatively,
$M_s \refinement M'_{s'}$.

A relation that satisfies {\bf atoms} and {\bf forth-$b$} for every $b \in A$,
and satisfies {\bf back-$b$} for every $b \in A - \{a\}$, for some $a \in A$, is
an $a$\textit{-simulation}. 

We call $M'_{s'}$ an $a$-simulation of $M_s$, and we call $M_s$ an
$a$-refinement of $M'_{s'}$. We write $M'_{s'} \simulation_a M_s$ to denote
this, or alternatively, $M_s \refinement_a M'_{s'}$.
\end{definition}

Whilst we have introduced simulations and refinements as relations between
models, we will often also refer to Kripke models as refinements or simulations
of other models. For example, if $M_s \refinement_a M'_{s'}$ then we call $M_s$
an $a$-refinement of $M'_{s'}$, and we call $M'_{s'}$ an $a$-simulation of
$M_s$.

\begin{definition}[Positive formulae]
A positive formula is defined by the following abstract syntax:
$$
\phi ::=    p \bnfalt 
            \neg p \bnfalt
            \phi \land \phi \bnfalt
            \phi \lor \phi \bnfalt
            \knows_a \phi
$$
\end{definition}

\begin{lemma}
Let $M_s$ and $M'_{s'}$ be Kripke models such that $M'_{s'} \refinement M_s$,
and let $\phi$ be a positive formula. If $M_s \entails \phi$ then $M'_{s'}
\entails \phi$.
\end{lemma}

% TODO - Citation (Simulation and refinement?) or proof

The significance of refinements is that the finite refinements of a Kripke model
are exactly the models that result from the execution of an action
model~\cite{french2009simulation}. We will use $a$-refinements to define the
semantics of the future event logic.
